\documentclass[preprint,10pt]{elsarticle}

\usepackage[format=plain]{caption}
\usepackage{graphicx}
\usepackage{tikz}
\usepackage{amsmath} 
\usepackage{amssymb}
\usepackage{amsthm}
\usepackage{units}   
\usepackage{float}   
\usepackage{graphicx}
\usepackage{amsfonts}
\usepackage{mathrsfs}
\usepackage[colorlinks]{hyperref}
\usepackage{framed}
\usepackage{url}
\usepackage{longtable}

\floatstyle{ruled}
\newfloat{progfram}{thp}{lop}
%-------------------------------------------
\newcommand{\sign}{\ensuremath{\mathrm{sign}}}
\newcommand{\diag}{\ensuremath{\mathrm{diag}}}
\newcommand{\trace}[1]{\ensuremath{\mathrm{trace}\left( #1 \right)}}
\newcommand{\norm}[1]{\ensuremath{\left\|#1\right\|_2^2}}
\newcommand{\func}[2]{\ensuremath{\mathrm{#1}\left( #2 \right)}}
\newcommand\eps \epsilon
\newcommand{\RR}{\ensuremath{\mathbb{R}}}
\newcommand{\N}{\ensuremath{\mathbb{N}}}
\newcommand{\real}{\ensuremath{\mathbb{R}}}
\newcommand{\cl}[1]{\ensuremath{\mathcal{#1}}}
\newcommand{\suppsize}[1]{\ensuremath{|\mathcal{#1}|}}
\newcommand{\vect}[1]{\ensuremath{\mathbf{#1}}}
\newcommand{\matr}[1]{\ensuremath{\mathbf{#1}}}
\newcommand{\deriv}[2]{\frac{\partial #1}{\partial #2}}
\newcommand{\arr}[2]{\begin{array}{#1}#2\end{array}}
\newcommand{\mat}[2]{\left(\begin{array}{#1}#2\end{array}\right)}
\newcommand{\brc}[2]{\left\{\begin{array}{#1}#2\end{array}\right.}
\newcommand{\pars}[1]{\left(#1\right)}
\newcommand{\brcs}[1]{\left\{#1\right\}}
\newcommand{\half}{\frac{1}{2}}

% Bold symbols and operators
\newcommand\Laplacian{\nabla^2}
\newcommand\bnabla{\boldsymbol{\nabla}}
\newcommand\bLaplacian{\boldsymbol{\nabla}^2}
\newcommand\bcdot{\boldsymbol{\cdot}}
\newcommand\bU{\mathscr{\boldsymbol{U}}}
\newcommand\bv{\boldsymbol{v}}
\newcommand\bV{\boldsymbol{V}}
\newcommand\bE{\boldsymbol{E}}
\newcommand\be{\boldsymbol{\hat{e}}}
\newcommand\bn{\boldsymbol{\hat{n}}}
\newcommand\bk{\boldsymbol{\hat{k}}}
\newcommand\bj{\boldsymbol{j}}
\newcommand\bi{\boldsymbol{i}}
\newcommand\bs{\boldsymbol{s}}
\newcommand\bA{\boldsymbol{A}}
\newcommand\bF{\boldsymbol{F}}
\newcommand\bG{\boldsymbol{G}}
\newcommand\bI{\boldsymbol{I}}
\newcommand\bJ{\boldsymbol{J}}
\newcommand\bx{\boldsymbol{x}}
\newcommand\by{\boldsymbol{y}}
\newcommand\bz{\boldsymbol{z}}
\newcommand\br{\boldsymbol{r}}
\newcommand\bc{\boldsymbol{c}}
\newcommand\bxhat{\hat{\bx}}
\newcommand\byhat{\hat{\by}}
\newcommand\bzhat{\hat{\bz}}
\newcommand\brhat{\hat{\br}}
\newcommand\bnhat{\hat{\boldsymbol{n}}}
\newcommand\btheta{\boldsymbol{\theta}}
\newcommand\bthetahat{\hat{\btheta}}
\newcommand\bphi{\boldsymbol{\phi}}
\newcommand\bphihat{\hat{\bphi}}
\newcommand\bzero{\boldsymbol{0}}
\newcommand\bomega{\boldsymbol{\omega}}
\newcommand\bpsi{\boldsymbol{\psi}}
\newcommand\Du{\text{Du}}

% Calligraphic symbols
\newcommand\cB{\mathcal{B}}
\newcommand\cE{\mathcal{E}}
\newcommand\cF{\mathcal{F}}
\newcommand\cO{\mathcal{O}}
\newcommand\cG{\mathcal{G}}
\newcommand\cI{\mathcal{I}}
\newcommand\cP{\mathcal{P}}
\newcommand\cD{\mathcal{D}}
\newcommand\cL{\mathcal{L}}
\newcommand\cU{\mathscr{U}}
\newcommand\cV{\mathscr{V}}
\newcommand\cW{\mathscr{W}}

% Tensors
\newcommand\tI{\mathsf{I}}
\newcommand\tS{\mathsf{S}}
\newcommand\tT{\mathsf{T}}

% Unit vector
\newcommand\ui{\boldsymbol{\hat{\imath}}}


\journal{Computational Physics}

% For some reason, the hebrew package clashes with the amsthm package. If you need a proof environment, you can uncomment the following:
%\usepackage{amssymb} %Needed for \blacksquare. Take care not to add this package twice...
%\newenvironment{proof}[1][Proof]{\par \textbf{#1.} }{\hspace{10pt}\hfill$\blacksquare$\par}
\begin{document}

\begin{frontmatter}

\author{Roman Zeyde}
\ead{romanz@cs.technion.ac.il}

\author{Irad Yavneh}
\ead{irad@cs.technion.ac.il}

\address{Technion -- Israel Institute of Technology, Haifa 32000, Israel}

\title{Computational Electrokinetics}

\begin{abstract}
In this work we study the electrokinetic 
migration of particles in an electrolyte solution 
due to the application of an external electric field,
and propose a numerical framework for the iterative solution of such problems.
The electrokinetic transport process can be used for
transporting and manipulating micro- and nanoscale objects in 
many nanotechnology applications, nano-fluidic devices, packed-bed separation, 
desalination processes and various electrophoresis applications.
Due to strong electrostatic forces, a thin boundary layer forms near the particle
surface. This results in scale disparity of the boundary layer, which makes a 
full numerical solution challenging. 
We employ nonlinear macroscale effective boundary conditions that have been derived
using the specific chemical properties of the particle
(for an ion-exchanger particle by Yariv in \cite{yariv2010migration}
and for a surface charged inert particle by Schnitzer and Yariv in \cite{schnitzer2012surface}).
The resulting macroscale nonlinear partial differential system 
is discretized and an iterative Newton solver is
constructed automatically from the discrete equations.
Numerical results are obtained for an ion-exchanger and 
for a surface charged inert particle. 
An asymptotic analytical solution is 
used for the validation of the solver. 
The numerical results are compared to the asymptotic solutions, 
and good correspondence is achieved.
Finally, the new solver is applied in the regime of moderately large values of the external field, where no analytical results are known. The numerical results uncover the strong-field scaling behavior, including a new boundary layer at the front of the particle that scales exponentially with respect to the external field, and also a well-defined dependence of the particle steady-state velocity on this field.
\end{abstract}

\end{frontmatter}

\section{Introduction}

\subsection{The Physical Problem}

Electrokinetic theory describes the dynamics of charged particles
in ionic fluids \cite{masliyah2005book,kirby2010book}.
When a particle acquires surface charge, a layer
of ions of opposite charge is attracted to the surface via    
electric forces, creating a double-layer structure around the
particle (see Figure \ref{fig:EDL}). This structure, called
``Debye layer'', electrically screens the surface charge,
creating a potential difference between the particle and the outer
layer of the fluid bulk.
In cases where the layer width is much smaller than the particle
size, an analytical asymptotic solution for the Debye layer
dynamics can be derived \cite{yariv2010asymptotic}.
\begin{figure}
    \begin{center}
        \includegraphics[width=0.3\textwidth]
            {figs/debye.eps}
        \caption[Schematic structure of the double layer]{
        Schematic structure of the double layer:
        (1) charged particle surface, (2) Debye layer, (3) fluid bulk.
        If the particle surface has positive (black) surface charge,
        it attracts negative (white) ions from the fluid making the
        Debye layer negatively charged (as opposed to the rest of
        the fluid bulk, which is electrically neutral).
        The zeta potential is defined as the voltage drop across 
        the Debye layer (2).}
        \label{fig:EDL}
    \end{center}
\end{figure}
% \begin{tikzpicture}\end{tikzpicture}

The variables of the electrokinetic flow equations are the electrostatic
potential $\varphi$, fluid velocity $\bv$ and its pressure $p$, and
ionic concentration $c$.
The boundary conditions are determined by the specific
problem under consideration and defined by the particle's
geometry, chemical characteristics, and the fluid dynamics.
The partial differential equations that describe the system dynamics
under an external electric field, are coupled and nonlinear, and
in general have no analytic solution. 
Moreover, any numerical solver must handle the scale disparity caused by the
Debye layer width being much smaller than the particle size. 
An effective boundary condition outside the Debye layer
has been developed for spherical particles, yielding a macroscale 
model in \cite{yariv2010migration, schnitzer2012surface}.
The model is solved for weak electric field in an axisymmetric setting 
but it is hard to extend this analytic solution to more general systems.
Once the electric field becomes stronger, 
significant nonlinear phenomena emerge.
This interesting regime has not yet been explored extensively.

\subsection{Applications}

Recently, it has been conceived that such type of phenomena
can be very useful in microfluidics
devices and methods for controlling and manipulating fluid flows in micro- and nanoscale. 
Many different components are used in microfluidics such as valves, pumps, sensors, mixers, 
filters, separators, heaters etc., and are combined into lab-on-a-chip systems.

The manipulation can be achieved by either applying the external field at inlets and outlets, or it can be applied locally in the microchannel by integrated components.
A typical microchannel is produced with lithographic techniques. Normally the Reynolds
number is very low (on the order of unity or smaller) which means that viscosity dominates
and the convective term of the Navier-Stokes equations is insignificant. This also means that
microflows most often are laminar.

In the last decade there has been an increase in microfluidics research, 
due to the spread of tools for microfluidic systems fabrication, 
production of cheap and portable devices for fast medical analysis, 
and fundamental research of physical, chemical and biological processes.
Various applications are \cite{whitesides2006origins, erickson2004integrated, squires2005microfluidics, hardt2007microfluidic}:
DNA separation and sequencing \cite{wainright2003preconcentration},
cell analysis and forensic identification \cite{wu2004chemical, weinberger1993practical, horsman2007forensic},
integrated microchemical systems \cite{kothare2006dynamics},
packed-bed separation \cite{leinweber2004, losey2001microfabricated},
desalination processes \cite{kim2010direct},
drug delivery and pathogen detection \cite{santini2000microchips}.


Because the nonlinear macroscale model for these electrokinetic
phenomena has no closed-form analytic solution, a numerical
solver for this system is of importance.

%%%%%%%%%%%%%%%%%%%%%%%%%%%%%%%%%%%%%%%%%%%%%%%%%%%%%%%%%%%%%%%%%%%%%%%%%%%%%%%%%%%%%%%%%%%%%%%%%
\subsection{Governing Equations} \label{sec:equations}

\subsubsection{Nondimensionalization}
We follow the derivation presented by Yariv in \cite{yariv2010asymptotic}.
Spatial coordinates $\br$ are normalized by $a^*$ 
(the characteristic size of the particle).
Positive and negative ionic concentrations are denoted by $c_+$ and $c_-$, respectively, and
are normalized by the ambient ionic concentration $c^*$. The ionic valences are $\pm z$.
The dimensional ionic fluxes, $\bj_\pm$, are given by the Nernst-Plank equation 
(using the Einstein relation for the electric mobility, $\mu_q^* = D^* / k_B^* T^*$, 
where $D^*$ is the ionic diffusivity):
\begin{equation}
\bj^*_\pm = 
-D^* \bnabla c^*_\pm + \bv^* c^*_\pm \mp \frac{z e^* D^*}{k_B^* T^*} c^*_\pm \bnabla \varphi^*,
\end{equation}

The electric potential $\varphi$ is normalized by the thermal voltage:
$\varphi^* = {k_B^* T^*}/{z e^*} = {R^* T^*}/{z F^*},$
where $R^* = k_B^* N_A$, $F^* = e^* N_A$ is the Faraday number and $N_A$ is the Avogadro constant.

The dimensional Stokes equations with electrostatic forces are given by:
\begin{equation} \begin{array}{cc}
\bnabla \cdot \bv = 0, &
-\bnabla p^* + \mu^* \bLaplacian \bv^* + \eps^* \Laplacian \varphi^* 
\bnabla \varphi^* = 0.
\end{array}\end{equation}

The velocity $\bv$ is normalized by $v^* = {\eps^* (\varphi^*)^2}/{a^* \mu^*}$,
and the pressure $p$ is normalized by $p^* = {\mu^* v^*}/{a^*} = {\eps^* (\varphi^*)^2}/{(a^*)^2}$.
The fluxes $\bj^*_\pm$ are normalized by $j^* = {D^* c^*}/{a^*}$.

The dimensional Poisson equation can be written as
\begin{equation}
(c^*_+ - c^*_-) z e^* = -\eps^* \Laplacian \varphi^*,
\end{equation}

After nondimensionalization, the Nernst-Plank, Stokes and Poisson equations read:
\begin{equation} \label{eq:nernst}
\bj_\pm = 
-\bnabla c_\pm + \alpha \bv c_\pm \mp c_\pm \bnabla \varphi,
\end{equation}
\begin{equation} \label{eq:stokes}
\begin{array}{cc}
\bnabla \cdot \bv = 0, &
-\bnabla p + \bLaplacian \bv + \Laplacian \varphi \bnabla \varphi = 0, 
\end{array}
\end{equation}
\begin{equation} \label{eq:poisson}
c_+ - c_- = -2\delta^2 \Laplacian \varphi,
\end{equation}
where $\delta = {\delta^*}/{a^*}$ is the dimensionless Debye thickness 
($\delta^* = \sqrt{{\eps^* \varphi^*}/{2 c^* e^* z}}$) and 
$\alpha = {v^* a^*}/{D^*} = {\eps^* (\varphi^*)^2}/{\mu^* D^*}$ 
is the dimensionless Peclet number.
The force on the particle is denoted by $\bF$ normalized by $\mu^* v^* a^* = \eps^* (\varphi^*)^2$.
The stress tensor is denoted by $\tT$ and normalized by $p^* = \mu^* v^* / a^*$.
The electric field is denoted by $\bE$ and normalized by $E^* = \varphi^* / a^*$.
The surface charge is denoted by $\sigma$ and normalized by $\sigma^* = \eps^* \varphi^* / \delta^*$.

The ionic fluxes are given by ion diffusion, electrostatic forces and ion advection by the fluid
in \eqref{eq:nernst}. Due to conservation of ion concentrations, the fluxes are divergence-free:
\begin{equation*}
\bnabla \cdot \bj_\pm = 0.
\end{equation*}
These equations may be re-written by using salt $c = (c_+ + c_-)/2$ and charge $q = (c_+ - c_-)/2$.
Thus, salt and charge fluxes are defined by:
\begin{equation}\label{eq:fluxes}\begin{array}{cc}
  \bj = \frac{\bj_+ + \bj_-}{2} = -\bnabla c - q \bnabla \varphi + \alpha \bv c, & 
  \bi = \frac{\bj_+ - \bj_-}{2} = -\bnabla q - c \bnabla \varphi + \alpha \bv q,
\end{array}\end{equation}
and are divergence-free as well:
\begin{equation}\label{eq:zero_flux}\begin{array}{cc}
\bnabla \cdot \bj = 0, & 
\bnabla \cdot \bi = 0. 
\end{array}\end{equation}
In addition, the charge concentration is governed by the Poisson equation \eqref{eq:poisson}
\begin{equation}
q = -\delta^2 \Laplacian \varphi.
\end{equation}

\subsubsection{Boundary conditions}
We assume that a constant electric field of nondimensional magnitude $\beta$,
is applied to the system, in the direction denoted by the unit vector $\ui$.
As a result, the particle will equilibrate at 
a non-zero steady-state velocity $\cU \ui$, which is
a function of the applied field, and is not known a priori.

We employ the reference frame of the particle.
Far away from the particle, we have a uniform applied electrostatic field, uniform flow,
and ambient ionic concentrations:
\begin{equation}\label{eq:bndcond_inf}\begin{array}{ccc}
\bv \rightarrow -\cU \ui, &
\bnabla \varphi \rightarrow -\beta\ui, &
 c_\pm \rightarrow 1.
\end{array}\end{equation}

We consider a spherical particle.
The chemical properties of the particle's surface $\mathcal{S}$ (defined at $r=1$ 
with normal $\bn = \brhat$), determine the boundary conditions on $\mathcal{S}$.
We consider the following scenarios:
\paragraph{Ion-exchanger}
We follow the microscale model presented at \cite{yariv2010migration}
and assume high conductance of the particle (yielding a uniform electric potential),
zero slip, anion impermeability, cation selectivity 
(with fast Butler-Volmer\cite{bard2000book} kinetics
$\bn \cdot \bj_+ = k (1 - c_+/\gamma)$, where $k \rightarrow \infty$):
\begin{equation}
\begin{array}{cccc}
\varphi = \cV, &
\bv = \bzero, &
\bn \cdot \bj_- = 0, &
c_+ = \gamma,
\end{array}
\end{equation}
where $\gamma$ is the ratio between the cationic concentration on the surface and the cationic
concentration in the fluid bulk and $\cV$ is an unknown constant electric potential at
the ion-exchanger surface. 

\paragraph{Inert particle}
We follow the microscale model presented at \cite{schnitzer2012surface}
and assume constant surface charge
zero slip and ion impermeability:
\begin{equation}
\begin{array}{cccc}
\bn \cdot (\bnabla \varphi - \chi \bnabla \varphi_s) = \delta^{-1} \sigma, &
\bv = \bzero, &
\bn \cdot \bj_\pm = 0,
\end{array}
\end{equation}
where $\chi$ is the ratio between the dielectric constants of the 
particle and the fluid.

%%%%%%%%%%%%%%%%%%%%%%%%%%%%%%%%%%%%%%%%%%%%%%%%%%%%%%%%%%%%%%%%%%%%%%%%%%%%%%%%%%%%%%%%%%%%%%%%%%
\subsection{Separation of scales}
When $\delta \ll 1$, 
the fluid remains electroneutral ($q \approx 0$, thus $c_+ \approx c_-$), 
except inside a thin boundary layer of width $O(\delta)$ 
surrounding the particle (the ``Debye layer'').
This boundary layer makes it very hard to construct 
a direct numerical solver for the full electrokinetic problem.
However, the scale disparity at the Debye layer can be exploited
(as shown by Yariv in \cite{yariv2010asymptotic}), 
which allows the PDE system to be rewritten in the ``inner'' region 
as an ODE system using the asymptotic limit of $\delta \rightarrow 0$.
This system can be integrated and solved analytically,
yielding ``effective'' boundary conditions near
the particle surface for the ``outer'' bulk region equations.

\subsubsection  {Bulk-scale equations}
The Nernst-Planck equations above \eqref{eq:fluxes} for the fluid bulk outside
the boundary layer \cite{yariv2010asymptotic}, 
using bulk variables denoted by uppercase letters, read:
\begin{equation} \begin{array}{cccc} \label{eq:bulk_flux}
  Q = 0, &
  C = C_+ = C_-, &
\bJ = -\bnabla C + \alpha \bV C, &
\bI = -C \bnabla \varPhi.
\end{array}\end{equation}
Because the flow is incompressible ($\bnabla \cdot V = 0$), 
ion conservation results in the following 
salt and charge conservation equations -- the salt and charge fluxes above 
\eqref{eq:bulk_flux} are divergence-free
\eqref{eq:zero_flux}:
\begin{equation} \begin{array}{ccc}
\label{eq:salt_charge}
\Laplacian C - \alpha \bV \cdot \bnabla C = 0, &
\bnabla \cdot \pars{ C \bnabla \varPhi } = 0.
\end{array}\end{equation}
The macroscale Stokes flow equations have the same form as before \eqref{eq:stokes}:
\begin{equation}
\begin{array}{ccc}
\bnabla \cdot \bV = 0, &   
\Laplacian \bV - \bnabla P + \bnabla \varPhi \Laplacian \varPhi = \bzero.
\end{array}
\end{equation}

\subsection{Bulk-scale effective boundary conditions}
By integrating the equations across the Debye layer,
the effective boundary conditions at the particle surface are:
\subsubsection{Ion-exchanger} See \cite{yariv2010migration} for full derivation
of the nonlinear effective boundary conditions:
\begin{align}
\label{eq:ionex_bnd}
0 = \varPhi + \log C, 0 = \deriv{}{n} \pars{\varPhi - \log C}, 
\bV = 
\zeta \cdot \bnabla_\mathcal{S} \varPhi 
+ 2\log\pars{1-\tanh^2\frac{\zeta}{4}} \cdot \bnabla_\mathcal{S} \log C,
\end{align}
where $\zeta = \cV - \varPhi = \log (C / \gamma)$ 
is the zeta potential.

\subsubsection{Highly charged inert particle}
See \cite{schnitzer2012surface} for full derivation
of the nonlinear effective boundary conditions. 
Note that due to surface conduction, the resulting boundary conditions are:
\begin{align} \label{eq:ephor_bnd}
0 = {\deriv{C}{n} - \Du \nabla^2_s \pars{\varPhi - \log C}}, 
0 = {\deriv{C}{n} + C \deriv{\varPhi}{n}}, 
\bV = \zeta \cdot \bnabla_\mathcal{S} \pars{\varPhi - \log C} +
4 \log 2 \cdot \bnabla_\mathcal{S} \log C,
\end{align}
where $\zeta = \bar{\zeta} - \log C$ 
is the zeta potential and $\bar{\zeta} = 2 \log \sigma$.

\subsection{Steady-state velocity}
The stress tensor in the fluid is composed of Newtonian and Maxwel stresses:
\begin{equation}
\label{eq:tensor}
\tT = \bnabla \bV + (\bnabla \bV)^\dagger - P \tI
+ \bnabla \varPhi \bnabla \varPhi - \frac{1}{2} (\bnabla \varPhi \cdot \bnabla \varPhi). 
\end{equation} 
In steady state (constant particle drift velocity $\cU$), the total force acting on the particle 
(computed by the following surface integral) must vanish:
\begin{equation} \label{eq:zero_force}
 \bF(\beta, \cU) = \oint_\mathcal{S} \tT \cdot \bnhat dA = \bzero.
\end{equation}

The total force $\bF(\beta, \cU)$ is a function of the applied electric field $\beta$ and
the velocity $\cU$.
For any given electric field $\beta$, the force-free constraint above 
can be solved numerically for $\cU$ to yield the corresponding steady-state velocity.

\subsection{Objective of this work}

The goal of this work is to develop and implement a software toolbox 
for the generation of iterative numerical solvers for nonlinear macroscale 
electrokinetic problems, and to apply it to the study
of systems that have no closed-form solutions, 
such as ion-exchanger migration \cite{yariv2010migration} 
and the electrophoresis problem \cite{schnitzer2012surface}.
This solver can be used to gain insight into the chemical 
and physical behavior in far more general regimes than 
are currently well-understood, 
and also to possibly lead to further analytical developments, 
based on new scaling behavior
discovered numerically. 
We may also target interesting physical
phenomena that have been observed experimentally.
One such phenomenon is nonlinear electrokinetic vortex flow, 
which is useful for microfluidic mixing applications 
\cite{wang2004mix, ben2002vortex}.


%%%%%%%%%%%%%%%%%%%%%%%%%%%%%%%%%%%%%%%%%%%%%%%%%%%%%%%%%%%%%%%%%%%%%%%%%%%%%%
\section{Numerical Scheme} \label{ch:algorithm}

\subsection{Algorithm Description} \label{sec:algorithm}
The numerical solver first requires constructing a discrete 
approximation for the nonlinear PDE system,
for a given electric field $\beta$ and assumed velocity $\cU$. 
The resulting nonlinear discrete system is
solved numerically using Newton's Method, so the total force on the
particle can be computed. 
We seek the steady-state velocity $\cU$, 
which corresponds to zero total force, $F(\beta, \cU) = 0$,
which we solve by the standard secant method.

\subsubsection{Spherical Coordinates}
Because the system is axisymmetric and the particle is a sphere, we employ
spherical coordinates $(r,\theta,\phi)$ (as described in the appendix).

Note that in the spherical coordinate system,
the vector Laplacian operator cannot be computed in a decoupled manner as in the Cartesian system,
because the unit vectors $\brhat, \bthetahat, \bphihat$ are functions of the coordinates
in a curvilinear coordinate system.

\subsubsection{Computational Grid}
A regular grid of size $n_r \times n_\theta$ is defined, with:
\begin{align} \label{eq:grid}
(r_i,\theta_j) \in [1, \infty) \times [0,\pi], 
r_i = (1+\Delta_r)^i, 
\theta_j = \Delta_\theta \cdot j,
\end{align}
where a logarithmic grid spacing has been chosen for $r$ and a uniform grid for $\theta$.
Note that $r_0 = 1$ and $R_{max} = (1+\Delta_r)^{n_r}$, where
$\Delta_r = \pars{R_{max}} ^ {1/n_r} - 1$ and $\Delta_\theta = {\pi/n_\theta}$.

Because the grid is finite, the choice of $R_{max}$ must be large enough 
so as to have a negligible effect on the solution. 
On the other hand, in order to have a high resolution grid near the particle surface,
$\Delta_r$ should be minimized. It should be noted that for a given grid size $n_r$,
these requirements cannot be satisfied simultaneously -- thus, a compromise is required
between large $R_{max}$ and small $\Delta_r$.

This grid induces a disjoint subdivision of the domain 
$\Omega = [1, R_{max}] \times [0,\pi] = \bigcup_{ij}\Omega_{ij}$ into cells.
A specific cell $\Omega_{ij}$ and its center $(\bar{r}_i, \bar{\theta}_j)$ are defined by \eqref{eq:grid}:
\begin{align}
\Omega_{ij} = [r_{i-1}, r_{i}] \times [\theta_{j-1}, \theta_{j}], 
\bar{r}_i = (r_{i-1} + r_{i})/2, 
\bar{\theta}_j = (\theta_{j-1} + \theta_{j})/2.
\end{align}

Each discrete variable is located with respect to its cell 
as shown at Figure \ref{fig:grids}.
$\varPhi$, $C$ and $P$ are represented by their value at the center of each cell, 
using an all-centered grid, whereas $\bV$ is represented by its values at cell 
boundaries, using a staggered grid:
\begin{align}
\varPhi^h_{[i,j]} = \varPhi(\bar{r}_i, \bar{\theta}_j), 
C^h_{[i,j]} = C(\bar{r}_i, \bar{\theta}_j), 
P^h_{[i,j]} = P(\bar{r}_i, \bar{\theta}_j), 
V_r^h{}_{[i,j]} = V_r(r_i, \bar{\theta}_j), 
V_\theta^h{}_{[i,j]} = V_\theta(\bar{r}_i, {\theta}_j).
\end{align}
\begin{figure}[h]
    \begin{center}
	\includegraphics[width=0.5\textwidth]{figs/grids.eps}
        \caption[Computational grid]{The computational grid is defined using
        a spherical coordinate system, due to axisymmetry of the physical problem
        and boundary conditions discretization on the particle surface.}
    \label{fig:grids}
    \end{center}
\end{figure}

\subsection{Operator Discretization} 
A finite-volume method with linear interpolation is used for flux discretization. 
Define the following discrete central difference operators:
\begin{align} \label{eq:disc_diff}
\cD_r(f^h){}_{\left[i+\half,j\right]} = \frac{f^h_{\left[i+1,j\right]} - f^h_{\left[i,j\right]}}
                       {r_{\left[i+1,j\right]} - r_{\left[i,j\right]}}, 
\cD_\theta(f^h){}_{\left[i,j+\half\right]} = \frac{f^h_{\left[i,j+1\right]} - f^h_{\left[i,j\right]}}
					   {\theta_{\left[i,j+1\right]} - \theta_{\left[i,j\right]}}.
\end{align}

Define the following linear interpolation operators:
\begin{align} \label{eq:disc_interp}
\cI_r(f^h)_{\left[i+\half,j\right]} = 
\frac{r_{\left[i+\half,j\right]} - r_{\left[i,j\right]}}
{r_{\left[i+1,j\right]} - r_{\left[i,j\right]}} 
f^h_{\left[i+1,j\right]} 
+ 
\frac{r_{\left[i+1,j\right]} - r_{\left[i+\half,j\right]}}
{r_{\left[i+1,j\right]} - r_{\left[i,j\right]}}
f^h_{\left[i,j\right]},
\\
\cI_\theta(f^h)_{\left[i,j+\half\right]} =
\frac{r_{\left[i,j+\half\right]} - r_{\left[i,j\right]}}
{r_{\left[i,j+1\right]} - r_{\left[i,j\right]}}
  f^h_{\left[i,j+1\right]} + 
\frac{ r_{\left[i,j+1\right]} - r_{\left[i,j+\half\right]}}
{r_{\left[i,j+1\right]} - r_{\left[i,j\right]}}
  f^h_{\left[i,j\right]}.
\end{align}

If the total flux of a cell is equal to zero ($\bnabla \cdot \boldsymbol{f} = 0$), 
then using \eqref{eq:disc_diff}, we have
\begin{eqnarray}
\frac{1}{r^2} \cD_r\pars{f^h_r r^2} + 
\frac{1}{r \sin\theta} \cD_\theta\pars{f^h_\theta \sin\theta} = 0. 
\end{eqnarray}

\subsection{System Equations} \label{sec:disc_equations}
Charge and salt fluxes \eqref{eq:bulk_flux} are discretized on grid cell boundaries 
(using an upwind scheme $\mathcal{U}$ for numerical stability at large cell Peclet number)  
using \eqref{eq:disc_diff} and \eqref{eq:disc_interp}:
\begin{align} 
\nonumber
I^h_r = -\cI_r(C^h) \cdot \cD_r(\varPhi^h), &
I^h_\theta = -\cI_\theta(C^h) \cdot \frac{\cD_\theta(\varPhi^h)}{r}, \\
J^h_r = -\cD_r(C^h) + \alpha V^h_r \cdot \mathcal{U}^{\bV^h}_r (C^h), &
J^h_\theta = -\frac{\cD_\theta(C^h)}{r} + \alpha V^h_\theta \cdot \mathcal{U}^{\bV^h}_\theta (C^h), 
\\ \nonumber
 \mathcal{U}^{\bV^h}_r(C^h)[i,j] = C^h\left[i-\frac{\sign(V^h_r)}{2}, j\right], &
 \mathcal{U}^{\bV^h}_\theta(C^h)[i,j] = C^h\left[i, j-\frac{\sign(V^h_\theta)}{2}\right]. 
\end{align}
Mass flux $\bV$ is discretized on grid cells boundaries, 
using the staggered velocity grid.
Force components ($\bF = -\bnabla P + \bLaplacian \bV + \bnabla \varPhi \Laplacian \varPhi$) 
are discretized on the staggered velocity grid using the finite difference
method, where linear interpolation is used for the Coulomb force and 
for vector Laplacian components.
\begin{align}
\nonumber
F^h_r &= -\cD_r(P^h) 
          + \cL(V^h_r) - \frac{2}{r^2} V^h_r 
		  - \frac{2}{r^2 \sin\theta} \cI_r(\cD_\theta (V^h_\theta \sin\theta))
          + \cD_r(\varPhi^h) \cdot \cI_r(\cL(\varPhi^h)), \\
\nonumber
F^h_\theta &= -\frac{\cD_\theta(P^h)}{r} 
		  + \cL(V^h_\theta) - \frac{F^h_\theta}{r^2 \sin^2\theta} 
		  + \frac{2}{r^2} \cI_\theta(\cD_\theta(F^h_r))
		  + \frac{\cD_\theta(\varPhi^h)}{r} \cdot \cI_\theta(\cL(\varPhi^h)), \\
\cL(f^h) &= \frac{1}{r^2}\cD_r\pars{\cD_r(f^h) r^2} + 
\frac{1}{r^2 \sin\theta} \cD_\theta\pars{\cD_\theta(f^h) \cdot \sin\theta}.
\end{align}


\subsection{Boundary Conditions} \label{sec:disc_boundary}

Far away from the particle boundary \eqref{eq:bndcond_inf}, we have:
\begin{equation}
\begin{array}{cccc}
\cD_r \pars{\varPhi^h} = -\beta \cos\bar{\theta}, &
\cI_r \pars{C^h} = 1 &
V_r^h = -\cU \cos\theta, &
V_\theta^h = \cU \sin\theta.
\end{array}\end{equation}
For $\theta = 0$ and $\theta = \pi$, the ``ghost'' points are defined by axial 
symmetry considerations:
\begin{equation} 
\begin{array}{cccc}
\cD_\theta \pars{\varPhi^h} = 0, &
\cD_\theta \pars{C^h} = 0, &
\cD_\theta \pars{V_r^h} = 0, &
V_\theta^h = 0
\end{array}
\end{equation}
The specific boundary conditions at $r=1$
depend on the chemical properties of particle surface.

\subsubsection{Ion-exchanger} The boundary conditions \eqref{eq:ionex_bnd} 
are implemented by:
\begin{equation}
\begin{array}{lll}
0 = \cI_r(\varPhi^h + \log C^h), & 0 = \cD_r(\varPhi^h - \log C^h), \\
V^h_r = 0, V^h_\theta = 4\log\pars{\frac{1}{2}\pars{1 + \exp\left\{\cI_\theta(\zeta^h)/2\right\}}} \cdot \cD_\theta(\zeta^h), &
\zeta^h = - \log \gamma - \cI_r(\varPhi^h)\\.
\end{array}
\end{equation}

\subsubsection{Highly charged inert particle} The boundary conditions \eqref{eq:ephor_bnd} 
are implemented by:
\begin{equation}
\begin{array}{lll}
0 = \cD_r(C^h) - \Du \cdot 
\pars{\cD_\theta \pars{\sin\theta \cdot \cD_\theta \cI_r\pars{\varPhi^h - \log C}}}/{\sin\theta}, &
0 = \cD_r(C^h) + \cI_r(C^h) \cD_r(\varPhi^h), \\
V^h_r = 0, 
V^h_\theta = \cI_\theta(\zeta^h) \cdot \cD_\theta(\cI_r \varPhi^h - \log \cI_r C^h) + 
4 \log 2 \cdot \cD_\theta(\log \cI_r C^h), 
& \zeta^h = \bar{\zeta} - \log \cI_r C.
\end{array}
\end{equation}

%%%%%%%%%%%%%%%%%%%%%%%%%%%%%%%%%%%%%%%%%%%%%%%%%%%%%%%%%%%%%%%%%%%%%%%%%%%%%%%%
\subsection{Solver Design}

\subsubsection{System Equations}

Let $\cO : \RR^N \rightarrow \RR^N $ be a nonlinear operator.
In order to solve the equation $\cO(\bx) = \bzero$, Newton's method is applied, 
given an initial solution $\bx_0$.
Each step requires the computation of the residual vector, $\br = \cO(\bx)$, 
the gradient matrix of the operator, $\bG = \bnabla \cO(\bx)$
(given the current solution $\bx$)
and the solution of the sparse linear system $\bG \Delta \bx = -\br$.

The system variable $\bx$ is defined as the concatenation $\bx = [\,\varPhi, C, V_r, V_\theta, P\,]$.
Thus, each problem variable can be computed by applying an appropriate projection operator on $\bx$.
The system can be written as a nonlinear operator $\cO(\bx) = \bzero$ 
acting on the system variable ${\bx}$, 
where $\cO$ is the system equations operator, 
derived by concatenation of the system equations (described in Subsection \ref{sec:disc_equations})
and the boundary conditions operators (described in Subsection \ref{sec:disc_boundary}).
This way, the same numerical solver handles both the nonlinear system equations and nonlinear
boundary conditions.

\subsubsection{Steady-state}
Let $\beta$ be a given nondimensional electric field.
Assume that $\cU \ui$ is the drift velocity of the particle 
(given that the fluid is at rest far away from it).
After the convergence of the solver, once the residual norm is below
a prescribed threshold $\|\cO(\bx_n)\| < \epsilon$, 
the total force acting on the particle  
can be computed by integrating $\tT \cdot \bn$ over the particle surface.

Due to symmetry considerations, the total force $\bF$ is 
aligned with $\ui$ and it vanishes iff $f_\imath = \ui \cdot \bF = 0$.
\begin{equation}
f_\imath = \pars{-P + 2\cD_r(V_r) + 
\frac{\pars{\cD_r(\varPhi)}^2}{2} - \frac{\pars{\cD_\theta(\varPhi)}^2}{2r^2}}\cos\theta 
 -\pars{\cD_r(V_\theta) - \frac{V_\theta}{r}
+ \frac{\cD_r(\varPhi)}{r} \cD_{\theta}(\varPhi)}\sin\theta.
\end{equation}
The integral above is approximated by 1D numerical quadrature
(using the mid-point rule)
\begin{equation}
F_\imath^h(\beta, \cU^h) = \sum_{i=1}^{n_\theta} f^h(\bar\theta_i) \cdot 
              2 \pi \sin\bar\theta_i \cdot \Delta\theta_i,
\end{equation}
to yield the total force $F_\imath^h(\beta, \cU^h)$ as a function of the drift velocity $\cU^h$.
Since the goal is to find the steady-state solution, $F_\imath^h$ 
is required to be zero --
and the appropriate $\cU^h$ is found by a simple 1D root-finding algorithm,
applied as an outer loop:
\begin{equation} \label{eq:disc_zero_force}
F_\imath^h(\beta, \cU^h(\beta)) = 0
\end{equation}

\subsubsection{Continuation}

The iterative solver above can in principle be used 
to compute the steady-state solution for any given $\beta$.
For $\beta \ll 1$, the linear terms are the dominant ones, 
so the solution is approximately linear in $\beta$ 
(as derived in \cite{yariv2010migration}), and the solver convergence is fast.
However, this is no longer true for $\beta \gtrsim 1$, because the nonlinear terms become dominant
and the iterative solver may not converge at all if we start with an arbitrary initial guess.
In order to find a solution for such $\beta$, a continuation method is used:
the solver is applied to a sequence of $\{\beta_i\}_{i=0}^n$ such that $\beta_0 = 0$,
$\beta_n = \beta$ and the solution $\bx_i$ for $\beta_i$ is used as the solver initializer
for the problem of $\beta_{i+1}$.

%%%%%%%%%%%%%%%%%%%%%%%%%%%%%%%%%%%%%%%%%%%%%%%%%%%%%%%%%%%%%%%%%%%%%%%%%%%%%%%%
\subsection{Solver Implementation}

Object-Oriented Design methodology is used for implementation.
The MATLAB programming language is chosen due to strong numerical capabilities
and high-level language features.
The source code of the solver is provided in \cite{source}.
MATLAB direct solver is applied to solve the sparse linear system for each Newton step,
using the backslash operator. 
The library used for solving the linear system is UMFPACK 5.0 \cite{davis2004umfpack}
(Unsymmetric MultiFrontal Sparse LU Factorization Package).

\subsubsection{Operator Interface} \label{sec:operator-interface}
The base \verb|Operator| interface is defined by 
supporting the computation of a residual vector \verb|Operator.res()|
and a gradient matrix \verb|Operator.grad()|, given an input vector $\bx$.
This interface is implemented by specific operator classes,
which are used to construct the system operator $\cO$, so that 
the Newton solver (that requires computing the residual and the gradient) 
can be applied automatically.
Each such operator may have other operators as its inputs, so the residual
and the gradient are computed recursively, using the chain rule (applied to operators).

\paragraph{Linear operator}
If $\cO$ is linear, it can be represented by a matrix $L$, such that:
$\cO(\bx) = L \bx$ and $\bnabla\cO = L$.
Note that finite difference $\cD$, interpolation $\cI$ and 
upwind selection $\mathcal{U}$ operators
can be implemented as linear sparse operators, by constructing 
an appropriate sparse matrix $L$, having $O(\dim \bx)$ non-zeroes.
This way, the computation of the residual is very fast, taking $O(\dim \bx)$ time,
and the gradient does not depend on $\bx$.

\paragraph{Pointwise scalar function}
Let $f: \RR \rightarrow \RR$ be a differentiable 1D function, whose derivative is
denoted by $f': \RR \rightarrow \RR$. The operator $\cF$ can be defined to
represent a pointwise application of $f$:
$\cF(\bx) = [f(x_1); \ldots; f(x_n)]$ and $\bnabla\cF(\bx) = \diag\{f'(x_1), \ldots, f'(x_n)\}$.
$f$ is defined and differentiated automatically using the MATLAB symbolic toolbox.

\paragraph{Constant value}
A constant operator $\mathcal{C}$ is defined by the constant vector $\bc$ 
(which does not depend on $\bx$):
$\mathcal{C}(\bx) = \bc$ and $\bnabla\mathcal{C} = \bzero$.

\paragraph{Binary operators}
Let $\cO_1$ and $\cO_2$ be two operators, so that their pointwise addition, subtraction,
are defined as follows:
$(\cO_1 \pm \cO_2)(\bx) = \cO_1(\bx) \pm \cO_2(\bx)$ and
$\bnabla (\cO_1 \pm \cO_2)(\bx) = \bnabla\cO_1(\bx) \pm \bnabla\cO_2(\bx)$.
The point-wise multiplication of $\cO_1$ and $\cO_2$ is defined as follows: 
$(\cO_1 \cdot \cO_2)(\bx) = \cO_1(\bx) \cdot \cO_2(\bx)$ and 
$\bnabla (\cO_1 \cdot \cO_2)(\bx) = \diag(\cO_2(\bx)) \bnabla\cO_1(\bx)
                                  + \diag(\cO_1(\bx)) \bnabla\cO_2(\bx)$.

\paragraph{N-ary operators}
Given $N$ operators $\{\cO_i\}_{i=1}^N$, their concatenation $\cO$ is defined as:
$\cO(\bx) = [\cO_1(\bx); \ldots; \cO_N(\bx)]$ and  
$(\bnabla\cO)(\bx) = [(\bnabla\cO_1)(\bx); \ldots; (\bnabla\cO_N)(\bx)]$.

\subsubsection{Handling the Singularity} \label{sec:singular}
The electric potential $\varPhi$ (in electrophoresis case) and the pressure $P$ (in both cases) 
are defined up to an additive constant, since the system equations and the boundary conditions
contain only their derivatives.
In order to overcome the singularity of the linear system 
$\bG \Delta\bx = -\br$, we set the variable at 
a specific grid point to be $0$.
In order to retain a square system, we remove one conservation equation at this grid point too.
In the case of pressure $P$, we remove one mass conservation equation,
$\bnabla \cdot \bV = 0$, at this grid point. 
In the case of electric potential, we remove the charge conservation
equation, $\bnabla \cdot \pars{C \bnabla \varPhi} = 0$, at this grid point.
For $\br = \cO(\bx)$ and $\bG = \bnabla \cO(\bx)$, the new square linear system reads:
$\mathbb{R} \bG \mathbb{P} \cdot \delta\bx = -\mathbb{R} \br$,
where $\mathbb{P}$ matrix is used to remove the columns
and $\mathbb{R}$ matrix is used to remove the rows of $\bG$,
where the update step is $\Delta \bx = \mathbb{P} \cdot \delta \bx$.

\section{Results and Discussion} \label{ch:results}

The numerical solver is applied to the nonlinear problem of an ion-exchanger migration 
driven by an electric field \cite{yariv2010migration}
and the electrophoresis of highly charged surface inert particle 
\cite{schnitzer2012surface}. 
The solver is applied to a series of values of the electric field $\beta$, ranging 
from $\beta \ll 1$ to $\beta \sim 1$.
For each $\beta$ value, the Newton solver is initialized with the linear solution and is iterated $K$ times.
Secant method iteration is applied $M$ times to find the steady-state velocity 
by finding $\cU$ such that $F_\imath(\cU, \beta) = 0$, thus computing 
the steady-state velocity $\cU(\beta)$. 
The resulting $\cU(\beta)$ is plotted on a log-log scale, so that the asymptotic behaviour
of the solution may be examined for a wide range of $\beta$ values.
 
In order to validate the numerical solver, we employ 
the results of the asymptotic analysis of the nonlinear 
system (which are presented in \ref{sec:asymp}). 
The numerical results are compared 
to the asymptotic solutions, and a good correspondence is observed.

%%%%%%%%%%%%%%%%%%%%%%%%%%%%%%%%%%
\subsection{Ion-exchanger results}
For $\beta \ll 1$, the linear regime \cite{yariv2010asymptotic} is dominant.
However, when $\gamma \approx 1$, the cubic regime, where $\cU = O(\beta^3)$, 
dominates the linear one, even for weak electric fields. 
This phenomenon is confirmed both by the numerical
results and by asymptotic expansion of the nonlinear system (for $\alpha=0$).
The switch from the linear regime to the cubic regime occurs at 
an electric field value of $\beta = O(\sqrt{|\gamma - 1|})$.
The numerical results $\cU^h$, 
as well as the analytical ones \eqref{eq:cubic}, are given in Figure \ref{fig:IonExCubic}.
The numerical solver uses $N_r \times N_\theta = (400 \times 50)$, 
$R_{\max} = 10^7$ ($\Delta_r \approx 0.041$),
$K = 3$, $M = 5$ and $\beta = 10^{t}$ for $t = -2:0.1:0.5$.

The linear solution for $\gamma \approx 1$, 
yields a sign change of the zeta potential $\zeta = \log (C/\gamma)$, when
$|\log\gamma| \approx 3\beta / 4$. This results in a sign change in the slip velocity,
which can be approximated for small $|\zeta| \ll 1$ as $V_\theta \approx \zeta \cdot \bnabla_S \varPhi$.
This sign change creates a vortex at the downstream end of the ion-exchanger, 
as demonstrated by the numerical solution and the asymptotic analysis results 
in Figures \ref{fig:Vortex1} and \ref{fig:Vortex2}.
Note that there is a good correspondence between the numerical
results and the analytical solution for $\beta < 1$, validating
the vortex formation.

\begin{figure}
    \begin{center}
    \includegraphics[width=0.33\textwidth]{figs/Cubic/ionex_cubic1.eps}
    \includegraphics[width=0.33\textwidth]{figs/Cubic/ionex_cubic2.eps}
        \caption[Ion exchanger steady-state velocity]{
        Steady-state velocity $\cU$ as a function of the 
        electric field magnitude $\beta$, for $\gamma < 1$ (left) and $\gamma > 1$ (right). 
        The analytic cubic solution \eqref{eq:cubic} is represented by curves
        (solid for $\cU > 0$, dashed for $\cU < 0$), 
        and the numerical results are represented by markers 
        (circles for $\cU > 0$, squares for $\cU < 0$). Note that linear and
        cubic regimes match well the numerical results, including the critical $\beta$ 
        behaviour \eqref{eq:crit_beta}. 
        }
	    \label{fig:IonExCubic}
    \end{center}
\end{figure}

\begin{figure}
    \begin{center}
	\includegraphics[width=0.33\textwidth]{figs/Cubic/Stream/N1.eps}
	\includegraphics[width=0.33\textwidth]{figs/Cubic/Stream/N2.eps}
	\includegraphics[width=0.33\textwidth]{figs/Cubic/Stream/N5.eps}
	\includegraphics[width=0.33\textwidth]{figs/Cubic/Stream/N10.eps}
        \caption[Ion exchanger streamlines -- numerical results]{
        Streamlines of the flow for various electric field $\beta$ values: 
        numerical results for $\gamma = 0.9$.  
        The steady-state velocity $\cU$ is positive, so the
        ion-exchanger drifts to the right. }
    \label{fig:Vortex1}
    \end{center}
\end{figure}

\begin{figure}
    \begin{center}
	\includegraphics[width=0.33\textwidth]{figs/Cubic/Stream/A1.eps}
	\includegraphics[width=0.33\textwidth]{figs/Cubic/Stream/A2.eps}
	\includegraphics[width=0.33\textwidth]{figs/Cubic/Stream/A5.eps}
	\includegraphics[width=0.33\textwidth]{figs/Cubic/Stream/A10.eps}
        \caption[Ion exchanger streamlines -- analytical results]{
        Streamlines of the flow for various electric field $\beta$ values: 
        analytical approximation for $\gamma = 0.9$ (see Section \ref{sec:asymp}).
        The steady-state velocity $\cU$ is positive, so the
        ion-exchanger drifts to the right.}
    \label{fig:Vortex2}
    \end{center}
\end{figure}

%%%%%%%%%%%%%%%%%%%%%%%%%%%%%%%%%%%%
\subsection{Electrophoresis results}
For $\beta \ll 1$, the linear regime \cite{schnitzer2012surface} is dominant.
In order to verify the solver, the numerical steady-state velocity is compared to the
analytical solution \cite{schnitzer2012cubic}, derived by Schnitzer and Yariv. 
Since the cubic correction is much smaller than
the linear solution for $\beta \ll 1$, the difference between numerical results and
the linear solution is compared to the cubic correction term. 

Because the leading order of the solution $\cU(\beta)$ is linear for $\beta \ll 1$, 
the numerical discretization error is $O(\beta)$ as well.
In order to reduce the discretization error of the numerical solution \eqref{eq:disc_zero_force}
and after verifying that the discretization error converges as $O(h^2)$ (Figure \ref{fig:quadratic}),
we employ Richardson extrapolation to $\cU^h$ and $\cU^{2h}$ using $\hat\cU = \frac{4}{3} \cU^h - \frac{1}{3} \cU^{2h}$.

The results of the extrapolation $\hat\cU$ 
and the analytical results 
are given at Figures \ref{fig:Electrophoresis1} and \ref{fig:Electrophoresis2}.
The numerical solver uses $R_{\max} = 10^2$,
$K = 3$, $M = 5$ and $\beta = 0.1 \cdot 2^{t}$ for $t = -2:0.5:5$.
The solver is applied to 2 grids of $N_r \times N_\theta = (128 \times 128)$ as the coarse grid 
(with $\Delta_r \approx 0.037$), and $N_r \times N_\theta = (256 \times 256)$ as the fine grid
(with $\Delta_r \approx 0.018$), in order to use Richardson extrapolation for $\hat\cU$.

\begin{figure}
    \begin{center}
    \includegraphics[width=.45\textwidth]{figs/quadratic_order.eps}
        \caption[Quadratic convergence of steady-state velocity]{
        The discretization error between the numerical and analytical solutions for 
        $\alpha = 0$, $\Du = 0.5$, $\bar\zeta = 6$ and $R_{max} = 100$. The markers represent
        the discretization errors for various grid sizes, and the dotted lines correspond to linear
        functions of $\beta$, scaled by $4^n$. 
        The graph shows that, as the number of grid points in each dimension increases by a factor of 2, 
        the discretization error decreases by a factor of 4.
        }
	    \label{fig:quadratic}
    \end{center}
\end{figure}

\begin{figure}
    \begin{center}
    \includegraphics[width=0.33\textwidth]{figs/Electrophoresis/numerical_vs_analytical_alpha=0.0_Du=0.5_zeta=6.0_Rmax=100.eps}
    \includegraphics[width=0.33\textwidth]{figs/Electrophoresis/departure_from_linear_alpha=0.0_Du=0.5_zeta=6.0_Rmax=100.eps}
        \caption[Highly-charged surface particle steady-state velocity]{
        Steady-state velocity $\cU$ as a function of the 
        electric field magnitude $\beta$, for highly-charged surface particle
        electrophoresis (with $\alpha = 0, \text{Du} = 0.5, \bar\zeta = 6$).
        The analytic solution velocity is represented by solid lines, 
        and the numerical results are represented by markers.
        The left figure shows the full numerical solution 
        compared to the analytical one \eqref{eq:cubic}, and
        the right figure compares the analytic cubic correction \cite{schnitzer2012cubic} 
        to the difference between the numerical results and the linear regime.}
	    \label{fig:Electrophoresis1}
    \end{center}
\end{figure}

\begin{figure}
    \begin{center}
    \includegraphics[width=0.33\textwidth]{figs/Electrophoresis/numerical_vs_analytical_alpha=0.5_Du=1.0_zeta=10.0_Rmax=100.eps}
    \includegraphics[width=0.33\textwidth]{figs/Electrophoresis/departure_from_linear_alpha=0.5_Du=1.0_zeta=10.0_Rmax=100.eps}
        \caption[Highly-charged surface particle steady-state velocity]{
        Steady-state velocity $\cU$ as a function of the 
        electric field magnitude $\beta$, for highly-charged surface particle
        electrophoresis (with $\alpha = 0.5, \text{Du} = 1, \bar\zeta = 10$).
        The analytic solution velocity is represented by solid lines, 
        and the numerical results are represented by markers.
        The left figure shows the full numerical solution 
        compared to the analytical one \eqref{eq:cubic}, and
        the right figure compares the analytic cubic correction \cite{schnitzer2012cubic} 
        to the difference between the numerical results and the linear regime.
        When $\beta$ is large, the analytical cubic solution loses its validity 
        because higher-order terms become significant.}
	    \label{fig:Electrophoresis2}
    \end{center}
\end{figure}

%%%%%%%%%%%%%%%%%%%%%%%%%%%%%%%%%%%%%%%%%%
\subsection{Electrophoresis for large $\beta$ }
When the electric field is large, nonlinear effects dominate the solution.
In order to investigate the solution for strong electric fields, 
we have run the solver on the TAMNUN HPC cluster \cite{tamnun}.
Large grid sizes are used to verify that the numerical solution converges when the grid becomes finer and that no numerical artifacts are present in the solution.
Moreover, several $R_{max} \in \{10, 30, 100\}$ values are used, to make sure it has no major impact on the solution.
The results are shown in the following figures. 
The steady-state velocity results are shown for several grid resolutions and several $R_{max}$ values in Figure \ref{fig:LargeBetaV_grids}, and the differences between the numerical results for
steady-state velocity and the linear regime are shown in Figure \ref{fig:LargeBetaV_DeltaFromLinear_grids}.
The estimated rate of convergence is shown in Figure \ref{fig:LargeBetaRatio}.

The numerical solver converges well for $\beta < 5$ and the resulting steady-state velocity 
for moderate $\beta$ values is very close to being an affine function of $\beta$.
When $\beta$ becomes large, the salt concentration $C$ near the front of the particle 
(at $\theta \approx 0$) approaches zero, while creating a wake behind the particle 
(at $\theta \approx \pi$) as shown in Figure \ref{fig:LargeBeta_C}. 
This results in high gradients in $\log C$ near the front of the particle  and, therefore, a strong electric field $\bE$ forms near the particle surface (Figure \ref{fig:LargeBeta_Phi}) due to effective boundary conditions in (\ref{eq:ephor_bnd}). 
From Figs. \ref{fig:LargeBetaV_grids}-\ref{fig:LargeBetaV_DeltaFromLinear_grids} we conclude that even our most accurate numerical solution is valid only up to $\beta = 5$ approximately, and therefore we only further examine solutions in this range.

The streamline pattern does not change drastically for different electric field magnitudes. 
The resulting fluid flow is nearly proportional to that of the Stokes creeping flow, 
as shown in Figure \ref{fig:LargeBeta_Psi} and \ref{fig:LargeBeta_PsiMinusUniformFlow}. 
The difference between $\Psi(r, \theta)$ and $\Psi(r, \pi-\theta)$ is shown in Figure \ref{fig:LargeBeta_PsiAsymm}, demonstrating that the flow is almost symmetric (with relative
difference up to approximately $0.5\%$) for $\beta < 5$.

The numerical results show that a boundary layer starts to form near the front 
of the particle for strong electric fields (see Figures \ref{fig:LargeBeta_BoundaryLayer_E}
and \ref{fig:LargeBeta_BoundaryLayer_C}).
The width of the boundary layer with respect to the electric field scales, to a good approximation, as: $\bar\rho(\beta) \approx 0.2 e^{-0.5 \beta}$.
This causes numerical problems for the current solver at large $\beta$, due to insufficient
grid resolution at the boundary layer of $C$ and $\bE$.

\begin{figure}
    \begin{center}
    \includegraphics[width=.3\textwidth]{figs/LargeBetaV_[128_256_512]_Rmax=10.eps}
    \includegraphics[width=.3\textwidth]{figs/LargeBetaV_[512x512]_Rmax=10_30_100.eps}
        \caption[Large $\beta$ results for $\cU$]
{Large $\beta$ results for $\cU(\beta)$ ($\Du = 1$, $\bar\zeta = 10$, $\alpha = 0.5$). 
The left figure shows the numerical results of $\cU(\beta)$ for $R_{max} = 10$ and the following grid sizes: $128 \times 128$, $256 \times 256$, $512 \times 512$. 
The right figure shows the numerical results of $\cU(\beta)$ for
$512 \times 512$ grid for $R_{max} = 100, 30, 10$. For small values of $\beta$, the linear
regime of $\cU(\beta)$ is dominant. 
For moderate $\beta$ values, the steady state velocity is very close to an affine function of $\beta$.
However, for larger values of $\beta$, the numerical solution
deviates significantly from the linear regime, due to convergence problems at $\beta > 6$, 
and for $\beta \sim 7$, the solver fails to converge. The numerical convergence 
depends on the grid resolution near the particle surface -- 
as it becomes finer (when $N_r$ increases or $R_{max}$ decreases), the solver convergence improves.}
	    \label{fig:LargeBetaV_grids}	    
    \end{center}
\end{figure}
\begin{figure}
    \begin{center}
    \includegraphics[width=.3\textwidth]{figs/LargeBetaV_DeltaFromLinear[128_256_512]_Rmax=10.eps}
    \includegraphics[width=.3\textwidth]{figs/LargeBetaV_DeltaFromLinear[512x512]_Rmax=10_30_100.eps}
        \caption[Large $\beta$ results for $\cU$ -- difference from linear regime]
{Large $\beta$ results for $\cU(\beta) - \beta\cU_1$ ($\Du = 1$, $\bar\zeta = 10$, $\alpha = 0.5$ and $\cU_1 \approx 4.258$ is the linear regime coefficient). 
The left figure shows the numerical results of $\cU(\beta)$ for $R_{max} = 10$ and the following grid sizes: $128 \times 128$, $256 \times 256$, $512 \times 512$. 
The right figure shows the numerical results of $\cU(\beta)$ for
$512 \times 512$ grid for $R_{max} = 100, 30, 10$.
It is seen that the difference is indeed nearly an affine function of $\beta$ in the large-$\beta$ regime, so long as the numerical solution remains sufficiently accurate.}
	    \label{fig:LargeBetaV_DeltaFromLinear_grids}	    
    \end{center}
\end{figure}
\begin{figure}
    \begin{center}
    \includegraphics[width=.33\textwidth]{figs/LargeBetaRatio.eps}
        \caption[Rate of convergence estimation for $\cU$ at large $\beta$]
        {Rate of convergence estimation for $\cU$ at large $\beta$  
        ($\Du = 1$, $\bar\zeta = 10$, $\alpha = 0.5$), 
        based on the numerical results of $\cU(\beta)$, for the grid sizes of 
        $128 \times 128$, $256 \times 256$, $512 \times 512$. 
        The rate of convergence $r(\beta)$ is estimated by 
        $r(\beta) \triangleq \left(\cU^{4h}(\beta) - \cU^{2h}(\beta)\right)/\left(\cU^{2h}(\beta) - \cU^{h}(\beta)\right)$.
        For small $\beta$, $r(\beta) \rightarrow 4$, showing 
        quadratic convergence for the linear regime. 
        For moderate $\beta$, the convergence rate slows down, 
        and it deteriorates for large $\beta$ values.
        }
	    \label{fig:LargeBetaRatio}
    \end{center}
\end{figure}

\begin{figure}
    \begin{center}
    \includegraphics[width=0.3\textwidth]{figs/LargeBeta/sol_beta=1.000e-01_[513x513]_Rmax=10.0_Du=1.00_zeta=10.00_alpha=0.50_C.eps}
    \includegraphics[width=0.3\textwidth]{figs/LargeBeta/sol_beta=8.000e-01_[513x513]_Rmax=10.0_Du=1.00_zeta=10.00_alpha=0.50_C.eps}
    \includegraphics[width=0.3\textwidth]{figs/LargeBeta/sol_beta=5.000e+00_[513x513]_Rmax=10.0_Du=1.00_zeta=10.00_alpha=0.50_C.eps}
        \caption[Electrophoresis results for $C$]
        {Electrophoresis results for $C$: grid size $512 \times 512$, 
        $R_{max} = 10$, $\Du = 1$, $\bar\zeta = 10$, $\alpha = 0.5$.
        
        For weak field (left), 
        the linear regime dominates the solution for 
        $C_\beta(r, \theta) \approx 1 - \beta \frac{\cos\theta}{2r^2}$.
        For moderate $\beta$ values (middle), the salt concentration starts
        to form a wake behind the particle ($\theta \approx \pi$).
        For large values of $\beta$ (right), there is a significant drop in
        salt concentration at the front of the particle -- 
        $C$ approaches zero at $\theta \approx 0$, such that 
        $\deriv{C}{r}$ becomes large 
        (since the change in $C$ happens on a small length scale).
        Due to the boundary conditions of the electrophoresis problem
        (given by (\ref{eq:ephor_bnd})), the electric field radial
        component given by 
        $E_r = -\deriv{\varPhi}{r} = -\frac{1}{C}\deriv{C}{r}$, 
        becomes singular where $C \rightarrow 0$.
        }
	    \label{fig:LargeBeta_C}	    
    \end{center}
\end{figure}

\begin{figure}
    \begin{center}
    \includegraphics[width=0.3\textwidth]{figs/LargeBeta/sol_beta=1.000e-01_[129x129]_Rmax=10.0_Du=1.00_zeta=10.00_alpha=0.50_PhiE.eps}
    \includegraphics[width=0.3\textwidth]{figs/LargeBeta/sol_beta=8.000e-01_[129x129]_Rmax=10.0_Du=1.00_zeta=10.00_alpha=0.50_PhiE.eps}
    \includegraphics[width=0.3\textwidth]{figs/LargeBeta/sol_beta=5.000e+00_[129x129]_Rmax=10.0_Du=1.00_zeta=10.00_alpha=0.50_PhiE.eps}
        \caption[Electrophoresis results for $\varPhi$]
        {Electrophoresis results for $\varPhi$:
        grid size $512 \times 512$, $R_{max} = 10$, $\Du = 1$, $\bar\zeta = 10$, $\alpha = 0.5$. 
        
        The electric potential and the electric field results are multiplied by $\beta^{-1}$,
	    so the linear regime appears unchanged for different values of $\beta$.
        For weak field (left), the linear regime dominates the solution for 
        $\varPhi_\beta(r, \theta) \approx -\beta {r \cos\theta}$,
        hence the electric field is constant for weak electric fields:
        $\bE \approx \beta \ui$.
        For moderate values of $\beta$ (middle),
        the symmetry breaks and the electric field becomes stronger 
        near the front of the particle than at the back.
	    For large values of $\beta$ (right), 
	    the electric field radial component $E_r = -\deriv{\varPhi}{r}$
	    at the particle front becomes much larger than the electric field at the particle back.
        }
	    \label{fig:LargeBeta_Phi}
    \end{center}
\end{figure}

\begin{figure}
    \begin{center}
    \includegraphics[width=.33\textwidth]{figs/LargeBeta/sol_beta=1.000e-01_[513x513]_Rmax=10.0_Du=1.00_zeta=10.00_alpha=0.50_[1]_Psi.eps}
    \includegraphics[width=.33\textwidth]{figs/LargeBeta/sol_beta=8.000e-01_[513x513]_Rmax=10.0_Du=1.00_zeta=10.00_alpha=0.50_[1]_Psi.eps}
    \includegraphics[width=.33\textwidth]{figs/LargeBeta/sol_beta=5.000e+00_[513x513]_Rmax=10.0_Du=1.00_zeta=10.00_alpha=0.50_[1]_Psi.eps}
    \includegraphics[width=.33\textwidth]{figs/LinearStokes_Psi.eps}
        \caption[Streamfunction results]
        {$\beta = 0.1, 0.8, 5.0$ and Stokes streamlines results: grid size $512 \times 512$, 
        $R_{max} = 30$, $\Du = 1$, $\bar\zeta = 10$, $\alpha = 0.5$.
        The numerical results for $\Psi$ are scaled by $\beta^{-1}$, so the linear regime
        will appear the same for different $\beta$ values.
        Note that the numerical results show a
        flow pattern very similar to that of the analytical Stokes flow.}
	    \label{fig:LargeBeta_Psi}	    
    \end{center}
\end{figure}

\begin{figure}
    \begin{center}
    \includegraphics[width=.33\textwidth]{figs/PsiMinusUniformFlow/sol_beta=1.000e-01_[513x513]_Rmax=10.0_Du=1.00_zeta=10.00_alpha=0.50_[1]_Psi.eps}
    \includegraphics[width=.33\textwidth]{figs/PsiMinusUniformFlow/sol_beta=8.000e-01_[513x513]_Rmax=10.0_Du=1.00_zeta=10.00_alpha=0.50_[1]_Psi.eps}
    \includegraphics[width=.33\textwidth]{figs/PsiMinusUniformFlow/sol_beta=5.000e+00_[513x513]_Rmax=10.0_Du=1.00_zeta=10.00_alpha=0.50_[1]_Psi.eps}
    \includegraphics[width=.33\textwidth]{figs/PsiMinusUniformFlow/LinearStokes_Psi-uniformflow.eps}
        \caption[Streamfunction results]
        {$\beta = 0.1, 0.8, 5.0$ and Stokes streamlines results after subtraction of the uniform
        flow streamfunction ($\frac{\cU}{2}r^2 \sin^2\theta$): grid size $512 \times 512$, 
        $R_{max} = 30$, $\Du = 1$, $\bar\zeta = 10$, $\alpha = 0.5$.
        The numerical results for $\Psi$ are scaled by $\beta^{-1}$, so the linear regime
        will appear the same for different $\beta$ values.
        Even for the largest beta we could compute, the velocity remains fairly symmetric and 		
        ``uneventful'' also in the framework of the ``laboratory'' rather than that of the particle.}
	    \label{fig:LargeBeta_PsiMinusUniformFlow}	    
    \end{center}
\end{figure}

\begin{figure}
    \begin{center}
    \includegraphics[width=0.33\textwidth]{figs/streamlines_asymmetry.eps}
        \caption[Streamfunction asymmetry]
        {Streamline results: grid size $512 \times 512$, 
        $R_{max} = 10$, $\Du = 1$, $\bar\zeta = 10$, $\alpha = 0.5$.
        The relative streamline asymmetry is computed by 
        $\|\Psi(r, \theta) - \Psi(r, \pi-\theta)\|/\|\Psi(r, \theta)\|$
        as a function of $\beta$. The results show that the streamline pattern is almost 
        symmetric, even for moderately large $\beta$ values.}
	    \label{fig:LargeBeta_PsiAsymm}	    
    \end{center}
\end{figure}

\begin{figure}
    \begin{center}
    \includegraphics[width=.33\textwidth]{figs/BoundaryLayerWidth_E1.eps}
    \includegraphics[width=.33\textwidth]{figs/BoundaryLayerWidth_E2.eps}
    \includegraphics[width=.33\textwidth]{figs/BoundaryLayerWidth_E3.eps}
    \includegraphics[width=.33\textwidth]{figs/BoundaryLayerWidth_E4.eps}
        \caption[Electric field boundary layer]
        {Electric field boundary layer formation: grid size $512 \times 512$, 
        $R_{max} = 30$, $\Du = 1$, $\bar\zeta = 10$, $\alpha = 0.5$. The upper left
        graph shows normalized electric field radial component $E_r$ near the particle front
        ($\theta = 0$). The numerical results for $e(\rho) = E_r(r=1+\rho, \theta=0)$
        are normalized by $\hat{e}(\rho) = \left(e(\rho) - \beta\right)/\left(e(0) - \beta\right)$,
        where $e(\rho = \infty) = \beta$.
        The normalized electric field satisfies $\hat{e}(\rho) \in [0, 1]$ and is shown as function
        of $\rho = r - 1$ at the upper left graph and as function of $\rho e^{0.5 \beta}$ at
        the upper right graph.
        In order to estimate the width of the boundary layer, for each $\beta$, we compute the value 
        of $\bar\rho(\beta) = \min_\rho \{ \rho : \hat{e}(\rho) < 0.5\}$. Each such $\bar\rho(\beta)$
        is plotted at the left graph as a solid dot on the corresponding line.
        In addition, $\bar\rho(\beta)$ is plotted as a function of $\beta$ using logarithmic 
        $y$ axis at the lower left graph to show that $\bar\rho(\beta) \approx 0.2 e^{-0.5 \beta}$.
        Moreover, $E_r(\beta; r=1, \theta=0) / \beta$ grows exponentially as a function of $\beta$,
        as shown at the lower right graph.
        Thus, a boundary layer is indeed forming near the front of the particle.
        }
	    \label{fig:LargeBeta_BoundaryLayer_E}	    
    \end{center}
\end{figure}


\begin{figure}
    \begin{center}
    \includegraphics[width=.33\textwidth]{figs/boundary_layer_C1.eps}
    \includegraphics[width=.33\textwidth]{figs/boundary_layer_C2.eps}
        \caption[Ionic concentration boundary layer]
        {Ionic concentration boundary layer formation: grid size $512 \times 512$, 
        $R_{max} = 30$, $\Du = 1$, $\bar\zeta = 10$, $\alpha = 0.5$. The left
        graph shows the ionic concentration $C$ near the particle front
        ($\theta = 0$). 
        In addition, $C_\beta(r=1, \theta=0)$ is plotted as a function of $\beta$ using logarithmic 
        $y$ axis at the right graph to show that 
        $C_\beta(r=1, \theta=0) \approx 0.94 e^{-0.45 \beta}$.
        The ionic concentration becomes very close to zero for $\beta \approx 7$ at $r = 1$, 
        creating a singularity near the particle front.
        }
	    \label{fig:LargeBeta_BoundaryLayer_C}	    
    \end{center}
\end{figure}

\begin{figure}
    \begin{center}
    \includegraphics[width=.33\textwidth]{figs/C_Er_product.eps}
        \caption{The product of $C(r=1, \theta=0) E_r(r=1, \theta=0) / \beta$ as
        a function of $\beta$.}
	    \label{fig:C_Er_product}	    
    \end{center}
\end{figure}


%%%%%%%%%%%%%%%%%%%%%%%%%%%%%%%%%%%%%%%%%%%%%%%%%%%%%%%%%%%%%%%%%%%%%%%%%%%%%%%%%%%%%%%%%%%5
\section{Discussion} \label{sec:discussion}
A numerical framework for a
nonlinear electrokinetic transport system is described.
The framework is used to construct numerical solvers 
for two electrokinetic problems: the migration of ion-exchangers
and surface-charged inert particle electrophoresis.
The numerical results for weak electric field 
exhibit a good correspondence to the asymptotic models as shown above.
Two interesting phenomena (cubic regime and downstream vortex) were predicted for the ion-exchanger
using the numerical results, and were verified using an asymptotic expansion of the nonlinear system.
The electrophoresis problem was solved on a high-resolution grid to validate the asymptotic 
cubic correction of the steady-state velocity, as well as to provide insights into
the solution behavior for moderately strong electric fields.
For strong electric field, the nonlinear effects become dominant and
a boundary-layer structure begins to form.
Due to diminishing ion concentration $C$ at the front of the particle,
a boundary layer of thickness approximately $0.2 e^{-0.5\beta}$ 
is formed with respect to the electric field $\bE$ at the front of the particle.
The value of $E_r / \beta$ at ($r = 1$, $\theta = 0$) is given approximately by 
$e^{\beta/2}$, while the value of $C(r = 1, \theta = 0)$ is given approximately 
by $e^{-\beta/2}$, and their product remains close to 1 for all $\beta$ that we could compute
(as shown in Figure \ref{fig:C_Er_product}).
However, the velocity of the particle remains nearly linear with $\beta$
and the velocity field is very close to being symmetric.


\section{Future Work}
Future work may include using the numerical results described in this work to develop
an analytical solution for the strong field regime of the electrophoresis problem, which is 
of interest. The solver may be used to investigate how the solution
(and the evolving boundary layer, in particular) depends on the problem parameters.
Optimizing the direct sparse linear solver can be done by
using the ordering data from previous calls,
or replacing the full Newton step by an iterative linear solver
(e.g., using preconditioned Krylov subspace methods \cite{saad2003book} or
Multigrid), allowing finer grids to be used.
In order to achieve high numerical accuracy, the grid discretization should be made finer
near the particle surface (at $r = 1$), while $R_{max}$ can be made smaller. Moreover, a different discretization scheme may be used for
the $r$ coordinate, to have higher grid resolution near the particles.
In addition, the continuation of the solution between different $\beta$ values
should advance in finer steps, to ensure Newton method convergence.
The discretization of boundary conditions far away from the particle may be improved
by including the linear correction from the asymptotic solution, or reformulation of
far-away boundary condition according to the asymptotic behavior of the analytic solution
far away from the particle.
Moreover, the framework may be used for automatic construction of Newton solvers 
for other nonlinear PDE systems, for example, for particles with different surface 
properties, or different particle shape.

\appendix

\section{Asymptotic Analysis} \label{sec:asymp}
In order to validate the numerical results, an asymptotic analysis has been applied 
to the nonlinear systems of ion-exchanger migration 
and highly charged particle electrophoresis.
We present here the asymptotic analysis for the ion-exchanger case \cite{yariv2010migration},
while the asymptotic analysis for the electrophoresis case \cite{schnitzer2012surface}
is derived by Schnitzer and Yariv in \cite{schnitzer2012cubic}.

The derivation is performed as follows.
The nonlinear system (composed of the governing equations and the boundary conditions) 
is written as $\cO(\bx;\beta) = \bzero$.
For small $\beta$, the solution to the nonlinear system can be expanded in a Taylor series in $\beta$:
\begin{equation}
\bx = \bx(\beta) \approx \sum_n \bx_n \beta^n.
\end{equation}
The nonlinear terms
are expanded around $\beta = 0$:
\begin{equation}
\bzero = \cO(\bx) = \sum_i \cO_i(\bx_0, \ldots \bx_i) \beta^i.
\end{equation}
Thus, the $O(\beta^k)$ term $\bx_k$ can be found recursively by solving 
$\cO_k(\bx_0, \bx_1, \ldots, \bx_k) = \bzero$,
given the previous solutions for $\bx_0, \ldots, \bx_{k-1}$.
The derivation is validated using the MATLAB symbolic toolbox by the code 
at \verb|symbolic/asymp.m| \cite{source}.

%%%%%%%%%%%%%%%%%%%%%%%%%%%%%%%%%%%%%%%%%%%%%%%%%%%%%%%%%%%%%%%%%%%%%%%%%%%
\subsection{First-Order (linear in $\beta$) Solution} \label{app:linear}

For $\beta = 0$ (no electric field is applied), the steady-state solution is $\cU = 0$:
\begin{equation}\begin{array}{cccc}
\varPhi_0(r,\theta) = 0, &
C_0(r,\theta) = 1, &
\bV_0(r,\theta) = \bzero, &
P_0(r,\theta) = 0.
\end{array}\end{equation}
Assuming $\beta \ll 1$, the linearized equations and boundary conditions are
\begin{equation} \begin{array}{ccc}
\Laplacian \varPhi_1 = 0, &
\Laplacian C_1 = 0, &
\bLaplacian \bV_1 - \bnabla P_1 = \bzero.
\end{array}\end{equation}
The first-order terms $\bx_1$ are given by (see \cite{yariv2010migration} for full derivation)
\begin{equation} \begin{array}{cccc}
\varPhi_1 = \pars{\frac{1}{4r^2} - r}\cos\theta, &
C_1 = \frac{3}{4r^2} \cos\theta, &
\Psi_1 = \cU_1 \pars{\frac{1}{r} - r^2} \frac{\sin^2\theta}{2}, &
P_1 = 0,
\end{array} \end{equation}
and the linear velocity term is:
\begin{align} \label{eq:linear_velocity}
\bV_1 &= \cU_1 \pars{ \brhat \pars{-1 + \frac{1}{r^3}}\cos\theta 
+ \bthetahat \pars{1 + \frac{1}{2r^3}} \sin\theta } \\
\cU_1(\gamma) &= 2 \log \pars{\frac{1 + \gamma^{-\frac{1}{2}}}{2}}.
\end{align}
For $\gamma > 1$, the Debye layer has a positive charge, 
corresponding to a negative particle charge -- 
yielding negative drift velocity, $\cU < 0$. 
For $\gamma < 1$, we have $\cU > 0$.

%%%%%%%%%%%%%%%%%%%%%%%%%%%%%%%%%%%%%%%%%%%%%%%%%%%%%%%%%%%%%%%%%%%%%%%%%%%
\subsection{Second-Order (quadratic in $\beta$) Solution} \label{app:quadratic}
The quadratic terms $\bx_2$ are computed by solving a linear PDE system 
with a right-hand side determined by the linear terms $\bx_1$, yielding
\begin{equation}
\varPhi_2 = \pars{\frac{\cU_1\, \alpha}{32} - \frac{1}{16}}
\frac{3\, {\cos^2\theta} - 1}{r^3} - \frac{3 + 3\, {\sin^2\theta}\, \left(4\, r^3 - 1\right)}{32\, r^4} - \frac{3\, \cU_1\, \alpha - 6}{32\, r},
\end{equation}
\begin{equation}
C_2 = \left(\frac{5\, \cU_1\, \alpha}{32} + \frac{1}{16}\right)
\frac{3\, {\cos^2\theta} - 1}{r^3} - \frac{3\, \cU_1\, \alpha - 6}{32\, r} + \frac{3\, \cU_1\, \alpha\, \left(2\, r^3\, {\sin^2\theta} + {\sin^2\theta} - 1\right)}{16\, r^4},
\end{equation}
\begin{equation}
\Psi_2 = \pars{\frac{9}{16(\sqrt{\gamma}+1)} - \frac{3}{16} \cU_1 (\cU_1 \alpha + 1)}
 \left(\frac{1}{r^2} - 1\right) \sin^2\theta \cos\theta.
\end{equation}
Note that the quadratic term does not contribute to the total force and steady-state 
velocity terms, but it does change the fluid flow, 
the electric potential and the ionic concentration.

%%%%%%%%%%%%%%%%%%%%%%%%%%%%%%%%%%%%%%%%%%%%%%%%%%%%%%%%%%%%%%%%%%%%%%%%%%%
\subsection{Third-Order (cubic in $\beta$) Solution} \label{app:cubic}
Note that, due to symmetry considerations, $\cU(\beta)$ is an anti-symmetric function.
Therefore, $\cU_2 = 0$ and the next velocity term is the cubic one:
\begin{equation} \label{eq:cubic}
\cU(\beta) \approx \beta \cU_1 + \beta^3 \cU_3 + O(\beta^5).
\end{equation}
The cubic terms $\bx_3$ are computed by solving a linear PDE system 
with a right-hand side determined by the linear terms $\bx_1$ and the quadratic terms $\bx_2$.

\begin{align*}
\varPhi_3 &= \frac{15\cU_1\alpha + 6 }{64}\cos\theta  - \frac{3 \cU_1\alpha}{32} {\cos}^3\theta
 + \frac{183 {\cU_1}^2\alpha^2 - 839 \cU_1\alpha - 470}{2560 r^2} \cos\theta 
\\ \nonumber &
+ \frac{3 \cos\theta \left(5 \cU_1\alpha - 6 {\cos^2\theta} - 4 \cU_1\alpha {\cos^2\theta} + 4\right)}{64 r^3} + \frac{\left(2 \cU_1\alpha - 1\right) \left(\cos\theta - 3 {\cos}^3\theta\right)}{64 r^5} 
\\ &+ {{\cos}^3\theta \left(\frac{15 \cU_1\alpha}{64} + \frac{3}{32}\right)}{r^{-2}} 
- \left(3 \cos\theta - 5 {\cos}^3\theta\right) \left( - \frac{19 {\cU_1}^2\alpha^2}{5120} + \frac{97 \cU_1\alpha}{5120} + \frac{3 \cU_2\alpha}{320} + \frac{21}{1280}\right){r^{-4}} 
\\ \nonumber &
+ {{\cos}^3\theta \left(\frac{\cU_1\alpha}{64} + \frac{3}{64}\right)}{r^{-6}},
\end{align*}
\begin{align*}
C_3 &=
\frac{3 {{\cU_1}}^2\alpha^2 {\cos^3\theta}}{32} + \frac{\cos\theta \left(797 {{\cU_1}}^2\alpha^2 + 419 {\cU_1}\alpha - 512 {\cU_2}\alpha + 174\right)}{2560 r^2} 
- \frac{3 {\cU_1}\alpha \cos\theta \left(5 {\cU_1}\alpha + 2\right)}{64} 
\\ \nonumber &
- {\left(3 \cos\theta - 5 {\cos^3\theta}\right) \left(\frac{59 {{\cU_1}}^2\alpha^2}{5120} + \frac{103 {\cU_1}\alpha}{5120} + \frac{69 {\cU_2}\alpha}{320} + \frac{3}{1280}\right)}{r^{-4}} 
\\ \nonumber &
+ \frac{\alpha \left(\cos\theta - 3 {\cos^3\theta}\right) \left(5\alpha {{\cU_1}}^2 + 2 {\cU_1} + 16 {\cU_2}\right)}{128 r^5} 
\\ \nonumber &
- \frac{3\alpha \cos\theta \left( - 8\alpha {{\cU_1}}^2 {\cos^2\theta} + 7\alpha {{\cU_1}}^2 + 2 {\cU_1} + 32 {\cU_2} {\cos^2\theta} - 32 {\cU_2}\right)}{128 r^3} 
\\ \nonumber &
+ \frac{{{\cU_1}}^2\alpha^2 {\cos^3\theta}}{32 r^6} - \frac{3 {\cU_1}\alpha {\cos^3\theta} \left(5 {\cU_1}\alpha + 2\right)}{64 r^2},
\end{align*}
\begin{align*}
\Psi_3 &=
r^2 {\sin^2\theta} \left(\frac{{\cW_2}}{15} - \frac{{\cW_1}}{3} + \frac{209}{3360}\right) 
 - {{\sin^2\theta} \left(\frac{5 {\cW_2}}{3} - \frac{{\cW_1}}{3} + \frac{35}{264}\right)}{r^{-1}}
 + {{\sin^4\theta} \left(2 {\cW_2} + \frac{761}{5632}\right)}{r^{-1}} 
\\ \nonumber & 
- \frac{{\sin^2\theta} \left(1848 r^6 {\sin^2\theta} - 2079 r^3 {\sin^2\theta} + 924 r^3 - 7 {\sin^2\theta} + 10\right)}{19712 r^5} 
%\\ \nonumber & 
\frac{4 {\sin^2\theta} - 5 {\sin^4\theta}}{r^3} \\ & + \left(\frac{{\cU_1}\alpha}{16} - \frac{1}{8}\right) \frac{2 r^3 - 3 r^2 + 1}{4 r} {\sin^2\theta},
\end{align*}
\begin{align*}
\cW_1 &= \frac{69}{512 \left(\sqrt{\gamma} + 1\right)} - \frac{1407\alpha^2 {\log\left(\frac{\sqrt{\gamma} + 1}{2 \sqrt{\gamma}}\right)}^3}{1280} - \frac{123 \log\left(\frac{\sqrt{\gamma} + 1}{2 \sqrt{\gamma}}\right)}{1280} - \frac{27}{512 {\left(\sqrt{\gamma} + 1\right)}^2} 
\\ & -\frac{1899\alpha {\log\left(\frac{\sqrt{\gamma} + 1}{2 \sqrt{\gamma}}\right)}^2}{2560} 
- \frac{\alpha \log\left(\frac{\sqrt{\gamma} + 1}{2 \sqrt{\gamma}}\right) \left(\frac{81}{16 \left(\sqrt{\gamma} + 1\right)} - \log\left(\frac{\sqrt{\gamma} + 1}{2 \sqrt{\gamma}}\right) \left(\frac{27\alpha \log\left(\frac{\sqrt{\gamma} + 1}{2 \sqrt{\gamma}}\right)}{4} + \frac{27}{8}\right)\right)}{320} 
\\ & + \frac{21\alpha \log\left(\frac{\sqrt{\gamma} + 1}{2 \sqrt{\gamma}}\right)}{128 \left(\sqrt{\gamma} + 1\right)},
\end{align*}
\begin{align*}
\cW_2 &= \frac{9 \log\left(\frac{\sqrt{\gamma} + 1}{2 \sqrt{\gamma}}\right)}{256} 
+ \frac{21\alpha^2 {\log\left(\frac{\sqrt{\gamma} + 1}{2 \sqrt{\gamma}}\right)}^3 + 
15\alpha {\log\left(\frac{\sqrt{\gamma} + 1}{2 \sqrt{\gamma}}\right)}^2}{64} 
- \frac{27}{512 {\left(\sqrt{\gamma} + 1\right)}^2}
\\ \nonumber &
- \frac{297\alpha \log\left(\frac{\sqrt{\gamma} + 1}{2 \sqrt{\gamma}}\right) + 54}{1024 \left(\sqrt{\gamma} + 1\right)}.
\end{align*}
The cubic solution satisfies the boundary conditions only when $\alpha\cU_1 = 0$,
due to boundary conditions mismatch for $C$ at $r \rightarrow \infty$, when
the diffusion and the advection terms become of the same order of magnitude.
The cubic correction velocity terms (for $\alpha = 0$)
turn out to be:
\begin{equation}
\cU_3(\gamma) = \frac{31}{320(\sqrt\gamma + 1)} - \frac{9}{320(\sqrt\gamma + 1)^2} + \frac{1}{1680} - \frac{11}{160} \log \pars{\frac{1 + \gamma^{-\frac{1}{2}}}{2}}.
\end{equation}
The cubic term becomes dominant for small $\beta$ when $\gamma \approx 1$.
Note that, for $\gamma = 1$, the linear term vanishes and the cubic term 
of the velocity is $\cU(\beta) = \frac{1129}{26880}\beta^3$.

Observe that a $\cU = 0$ solution may exist for a critical $\beta_c > 0$ and $\alpha = 0$, 
if it satisfies:
\begin{equation} \label{eq:crit_beta}
\beta_c^2 = -\frac{\cU_1(\gamma)}{\cU_3(\gamma)}.
\end{equation}
Specifically, for $\gamma = 1 + \eps$, where $0 < \eps \ll 1$, $\beta_c = O\pars{\sqrt{\eps}}$:
\begin{equation}
\beta_c \approx \sqrt{-{\cU_1'(1) \eps}/{\cU_3(1)}} = 
 \sqrt{{13440 \eps}/{1129}} \approx 3.45 \sqrt{\eps}.
\end{equation}
That is, there may be a specific nonzero external field for which the particle remains at rest. The physical viability of this solution has yet to be explored.

\clearpage
\addcontentsline{toc}{chapter}{Bibliography}

\bibliographystyle{unsrt}
\bibliography{thesis}

\end{document}
