\section{Asymptotic Expansion}

The non-linear system is defined by the following PDEs:
\label{PDEs}
\begin{eqnarray}
\bnabla \cdot \pars{C \bnabla \varPhi} &=& 0, \\
\bnabla \cdot \pars{\bnabla C - \alpha \bV C} &=& 0, \\
\bLaplacian \bV - \bnabla P + \Laplacian \varPhi \bnabla \varPhi &=& \bzero.
\end{eqnarray}

For $\zeta(\theta) = \cV - \varPhi(1, \theta)$ and 
$\cV = -\log \gamma$, the boundary conditions on $r = 1$ are:
\begin{eqnarray}
\varPhi + \log C &=& \cV + \log \gamma = 0, \\
\deriv{}{r}\pars{\varPhi - \log C} &=& \frac{J_-}{C} = 0, \\
\bV &=& 4\log\pars{\frac{1 + e^\frac{\zeta}{2}}{2}} \bnabla_S \varPhi.
\end{eqnarray}

The boundary conditions for $r \rightarrow \infty$ are:
\begin{eqnarray}
\bnabla \varPhi &=& -\beta \ui, \\
C &=& 1, \\
\bV &=& -\cU \ui,
\end{eqnarray}
and the total force acting on the particle is zero: $\bF = \bzero$.

The nonlinear system can be written as: 
\begin{eqnarray}
\cO(\bx;\beta) &=& \bzero.
\end{eqnarray}

The solution to the non-linear system can be written as a Taylor series in $\beta$.
\begin{eqnarray}
\bx = \bx(\beta) &\approx& \sum_n \bx_n \beta^n.
\end{eqnarray}

All the variables are expanded around $\beta = 0$:
\begin{eqnarray}
\varPhi &=& \beta \varPhi_1 + \beta^2 \varPhi_2 + \beta^3 \varPhi_3 + \ldots, \\
C &=& 1 + \beta C_1 + \beta^2 C_2 + \beta^3 C_3 + \ldots, \\
\Psi &=& \beta \Psi_1 + \beta^2 \Psi_2 + \beta^3 \Psi_3 + \ldots ,
\end{eqnarray}
where the velocity $\bV$ and pressure $P$ are computed from the streamfunction $\Psi$:
\begin{eqnarray}
\bV &=& \bnabla \times \pars{\frac{\Psi}{r \sin\theta} \bphihat} 
= \frac{1}{r^2 \sin\theta} \deriv{\Psi}{\theta} \brhat 
- \frac{1}{r \sin\theta} \deriv{\Psi}{r} \bthetahat.
\end{eqnarray}

The nonlinear equations and the boundary conditions are expanded up to the cubic term
and arranged by powers of $\beta$:
\begin{eqnarray}
\bzero = \cO(\bx) = \cO_0(\bx_0) 
   + \beta \cO_1(\bx_0, \bx_1) 
 + \beta^2 \cO_2(\bx_0, \bx_1, \bx_2) 
 + \beta^3 \cO_3(\bx_0, \bx_1, \bx_2, \bx_3) \ldots
\end{eqnarray}
This can be rewritten as a system of equations:
\begin{eqnarray}
\brc{rcl}{
\bzero &=& \cO_0(\bx_0), \\
\bzero &=& \cO_1(\bx_0, \bx_1), \\
\bzero &=& \cO_2(\bx_0, \bx_1, \bx_2), \\
& \vdots & \\
\bzero &=& \cO_k(\bx_0, \bx_1, \bx_2, \ldots \bx_k).
}
\end{eqnarray}
Thus, the $O(\beta^k)$ term $\bx_k$ can be found recursively by solving 
$\cO_k(\bx_0, \bx_1, \ldots, \bx_k) = \bzero$,
given the previous solutions for $\bx_0, \ldots, \bx_{k-1}$.

%%%%%%%%%%%%%%%%%%%%%%%%%%%%%%%%%%%%%%%%%%%%%%%%%%%%%%%%%%%%%%%%%%%%%%%%%%%
\subsection{Linear Solution $(k=1)$}

For $\beta = 0$ (no electric field is applied), the steady-state solution is $\cU = 0$:
\begin{eqnarray}
\varPhi_0(r,\theta) &=& 0, \\
C_0(r,\theta) &=& 1, \\
\bV_0(r,\theta) &=& \bzero, \\
P_0(r,\theta) &=& 0.
\end{eqnarray}

Assuming $\beta \ll 1$, the linearized equations and boundary conditions are:
\begin{eqnarray}
\Laplacian \varPhi_1 &=& 0, \\
\Laplacian C_1 &=& 0, \\
\bLaplacian \bV_1 - \bnabla P_1 &=& \bzero.
\end{eqnarray}

The first order terms are:
\begin{eqnarray}
\varPhi &=& \beta \pars{\frac{1}{4r^2} - r}\cos\theta, \\
C &=& 1 + \beta \frac{3}{4r^2} \cos\theta, \\
\Psi &=& \beta \cU_1 \pars{\frac{1}{r} - r^2} \frac{\sin^2\theta}{2}, \\
\bV &=& -\beta \cU_1 \brcs{\pars{1 - \frac{1}{r^3}}\cos\theta \cdot \brhat - 
                               \pars{1 + \frac{1}{2r^3}} \sin\theta \cdot \bthetahat}, \\
P &=& 0, \\
\bF &=& \bzero.
\end{eqnarray}

The slip velocity satisfies:
\begin{eqnarray}
\beta \cU_1 \frac{3}{2} \sin\theta =
V_\theta &=& 4 \log \pars{\frac{1 + e ^ \frac{\zeta}{2}}{2}} \deriv{\varPhi}{\theta} 
=
 3 \beta \log \pars{\frac{1 + e ^ \frac{\zeta}{2}}{2}} \sin\theta, \\
\cU_1 &=& 2 \log \pars{\frac{1 + e ^ \frac{\zeta}{2}}{2}} 
       =  2 \log \pars{\frac{1 + \gamma ^ {-\frac{1}{2}}}{2}} = \cW_1,
\end{eqnarray}
because $\varPhi_1 = -C_1 = -\frac{3}{4} \cos\theta$ on $r=1$
and $\zeta_0 = -\log\gamma$.

%%%%%%%%%%%%%%%%%%%%%%%%%%%%%%%%%%%%%%%%%%%%%%%%%%%%%%%%%%%%%%%%%%%%%%%%%%%
\subsection{High-order Solution}
Note that, due to symmetry considerations, $\cU(\beta)$ is an anti-symmetric function:
\begin{eqnarray}
\cU(-\beta) &=& -\cU(\beta).
\end{eqnarray}
Therefore, $\cU_2 = 0$ and the next velocity term is the cubic one:
\begin{eqnarray}
\cU(\beta) &\approx& \beta \cU_1 + \beta^3 \cU_3 + O(\beta^5)
\end{eqnarray}

The quadratic terms $\bx_2$ are computed by solving a linear PDE system 
with a right-hand side determined by the linear terms $\bx_1$, and using 
homogenous solutions to satisfy boundary conditions:
\begin{eqnarray}
\varPhi_2 &=& \frac{\left(\frac{\cU_1\, \alpha}{32} - \frac{1}{16}\right)\, \left(3\, {\cos^2\theta} - 1\right)}{r^3} - \frac{3}{32\, r^4} - \frac{3\, {\sin^2\theta}\, \left(4\, r^3 - 1\right)}{32\, r^4} - \frac{3\, \cU_1\, \alpha - 6}{32\, r},
\\
C_2 &=& \frac{\left(\frac{5\, \cU_1\, \alpha}{32} + \frac{1}{16}\right)\, \left(3\, {\cos^2\theta} - 1\right)}{r^3} - \frac{3\, \cU_1\, \alpha - 6}{32\, r} + \frac{3\, \cU_1\, \alpha\, \left(2\, r^3\, {\sin^2\theta} + {\sin^2\theta} - 1\right)}{16\, r^4},
\\
\Psi_2 &=& \cW_2
 \left(\frac{1}{r^2} - 1\right) \cos\theta \sin^2\theta,  \\
\cW_2 &=& \pars{\frac{9}{16(\sqrt{\gamma}+1)} - \frac{3}{16} \cU_1 (\cU_1 \alpha + 1)},\\
\bF_2 &=& \bzero, \\ \cU_2 &=& 0.
\end{eqnarray}

The cubic terms $\bx_3$ are computed  by solving a linear PDE system 
with a right-hand side determined by the linear and 
quadratic terms $\bx_1, \bx_2$, and using 
homogenous solutions to satisfy boundary conditions.

\begin{eqnarray*}
\varPhi_3 &=& \cos\theta \left(\frac{15 \cW_1\alpha}{64} + \frac{3}{32}\right) - \frac{3 \cW_1\alpha {\cos}^3\theta}{32} - \frac{\cos\theta \left( - 183 {\cW_1}^2\alpha^2 + 839 \cW_1\alpha + 470\right)}{2560 r^2} \\ 
&&+ \frac{3 \cos\theta \left(5 \cW_1\alpha - 6 {\cos^2\theta} - 4 \cW_1\alpha {\cos^2\theta} + 4\right)}{64 r^3} + \frac{\left(2 \cW_1\alpha - 1\right) \left(\cos\theta - 3 {\cos}^3\theta\right)}{64 r^5} 
\\ 
&&- \frac{\left(3 \cos\theta - 5 {\cos}^3\theta\right) \left( - \frac{19 {\cW_1}^2\alpha^2}{5120} + \frac{97 \cW_1\alpha}{5120} + \frac{3 \cW_2\alpha}{320} + \frac{21}{1280}\right)}{r^4} + \frac{{\cos}^3\theta \left(\frac{15 \cW_1\alpha}{64} + \frac{3}{32}\right)}{r^2} \\ 
&&+ \frac{{\cos}^3\theta \left(\frac{\cW_1\alpha}{64} + \frac{3}{64}\right)}{r^6},
\end{eqnarray*}
\begin{eqnarray*}
C_3 &=&
\frac{3 {{\cU_1}}^2\alpha^2 {\cos^3\theta}}{32} + \frac{\cos\theta \left(797 {{\cU_1}}^2\alpha^2 + 419 {\cU_1}\alpha - 512 {\cU_2}\alpha + 174\right)}{2560 r^2} \\
&&- \frac{\left(3 \cos\theta - 5 {\cos^3\theta}\right) \left(\frac{59 {{\cU_1}}^2\alpha^2}{5120} + \frac{103 {\cU_1}\alpha}{5120} + \frac{69 {\cU_2}\alpha}{320} + \frac{3}{1280}\right)}{r^4} \\
&&- \frac{3\alpha \cos\theta \left( - 8\alpha {{\cU_1}}^2 {\cos^2\theta} + 7\alpha {{\cU_1}}^2 + 2 {\cU_1} + 32 {\cU_2} {\cos^2\theta} - 32 {\cU_2}\right)}{128 r^3} - \frac{3 {\cU_1}\alpha \cos\theta \left(5 {\cU_1}\alpha + 2\right)}{64} \\
&&+ \frac{\alpha \left(\cos\theta - 3 {\cos^3\theta}\right) \left(5\alpha {{\cU_1}}^2 + 2 {\cU_1} + 16 {\cU_2}\right)}{128 r^5} + \frac{{{\cU_1}}^2\alpha^2 {\cos^3\theta}}{32 r^6} - \frac{3 {\cU_1}\alpha {\cos^3\theta} \left(5 {\cU_1}\alpha + 2\right)}{64 r^2},
\end{eqnarray*}
\begin{eqnarray*}
\Psi_3 &=&
r^2 {\sin^2\theta} \left(\frac{{\cU_{3b}}}{15} - \frac{{\cU_{3a}}}{3} + \frac{209}{3360}\right) 
 - \frac{{\sin^2\theta} \left(\frac{5 {\cU_{3b}}}{3} - \frac{{\cU_{3a}}}{3} + \frac{35}{264}\right)}{r}
\\ && + \frac{{\sin^4\theta} \left(2 {\cU_{3b}} + \frac{761}{5632}\right)}{r} - \frac{{\sin^2\theta} \left(1848 r^6 {\sin^2\theta} - 2079 r^3 {\sin^2\theta} + 924 r^3 - 7 {\sin^2\theta} + 10\right)}{19712 r^5} 
\\ && + \frac{\left(\frac{2 {\cU_{3b}}}{5} + \frac{829}{28160}\right) \left(4 {\sin^2\theta} - 5 {\sin\theta}^4\right)}{r^3} + \frac{{\sin^2\theta} \left(\frac{{\cU_1}\alpha}{16} - \frac{1}{8}\right) \left(2 r^3 - 3 r^2 + 1\right)}{4 r},
\\
\cU_{3a} &=& \frac{69}{512 \left(\sqrt{\gamma} + 1\right)} - \frac{1407\alpha^2 {\log\left(\frac{\sqrt{\gamma} + 1}{2 \sqrt{\gamma}}\right)}^3}{1280} - \frac{123 \log\left(\frac{\sqrt{\gamma} + 1}{2 \sqrt{\gamma}}\right)}{1280} - \frac{27}{512 {\left(\sqrt{\gamma} + 1\right)}^2} \\ && -
\frac{1899\alpha {\log\left(\frac{\sqrt{\gamma} + 1}{2 \sqrt{\gamma}}\right)}^2}{2560} - \frac{\alpha \log\left(\frac{\sqrt{\gamma} + 1}{2 \sqrt{\gamma}}\right) \left(\frac{81}{16 \left(\sqrt{\gamma} + 1\right)} - \log\left(\frac{\sqrt{\gamma} + 1}{2 \sqrt{\gamma}}\right) \left(\frac{27\alpha \log\left(\frac{\sqrt{\gamma} + 1}{2 \sqrt{\gamma}}\right)}{4} + \frac{27}{8}\right)\right)}{320} \\ && + 
\frac{21\alpha \log\left(\frac{\sqrt{\gamma} + 1}{2 \sqrt{\gamma}}\right)}{128 \left(\sqrt{\gamma} + 1\right)},
\\
\cU_{3b} &=& \frac{9 \log\left(\frac{\sqrt{\gamma} + 1}{2 \sqrt{\gamma}}\right)}{256} + \frac{21\alpha^2 {\log\left(\frac{\sqrt{\gamma} + 1}{2 \sqrt{\gamma}}\right)}^3}{64} - \frac{27}{512 \left(\sqrt{\gamma} + 1\right)} 
\\ && 
- \frac{27}{512 {\left(\sqrt{\gamma} + 1\right)}^2} + \frac{15\alpha {\log\left(\frac{\sqrt{\gamma} + 1}{2 \sqrt{\gamma}}\right)}^2}{64} 
- \frac{297\alpha \log\left(\frac{\sqrt{\gamma} + 1}{2 \sqrt{\gamma}}\right)}{1024 \left(\sqrt{\gamma} + 1\right)}.
\end{eqnarray*}

\subsection{Analysis}

\subsubsection{Advection-less regime}
The solution satisfies the $\lim_{r\rightarrow\infty}C_3 = 0$ boundary condition, 
when advection is absent:
\begin{eqnarray}
\alpha = 0.
\end{eqnarray}

In this case, the steady-state velocity is:
\begin{eqnarray}
\cU(\beta, \gamma) &=& \cU_1(\gamma) \beta + \cU_3(\gamma) \beta^3 \\ &=&
\beta \cU_1(\gamma) \pars{1 - \frac{11\beta^2}{320}} + 
\beta^3 \pars{\frac{31}{320(\sqrt\gamma + 1)} - \frac{9}{320(\sqrt\gamma + 1)^2} + \frac{1}{1680}}.
\end{eqnarray}
The velocity terms are:
\begin{eqnarray}
\cU_1(\gamma) &=& 2 \log \pars{\frac{1 + \gamma^{-\frac{1}{2}}}{2}}, \\
\cU_3(\gamma) &=& \frac{31}{320(\sqrt\gamma + 1)} - \frac{9}{320(\sqrt\gamma + 1)^2} + \frac{1}{1680} - \frac{11 \cU_1(\gamma)}{320}.
\end{eqnarray}

\subsubsection{Cubic regime}
Note that, for $\gamma = 1$, the steady-state velocity leading term is $O(\beta^3)$:
\begin{eqnarray}
\cU(\beta) = \frac{1129}{26880}\beta^3.
\end{eqnarray}

\subsubsection{Stationary solution}
Note that a $\cU = 0$ solution may exist for $\beta_c \ne 0$ and $\alpha = 0$, 
if it satisfies:
\begin{eqnarray}
\beta_c^2 = -\frac{\cU_1(\gamma)}{\cU_3(\gamma)}.
\end{eqnarray}
For $\gamma = 1 + \eps$, where $0 < \eps \ll 1$:
\begin{eqnarray}
\beta_c \approx \sqrt{-\frac{\cU_1'(1) \eps}{\cU_3(1)}} = 
 \sqrt{\frac{13440 \eps}{1129}} \approx 3.45 \sqrt{\eps}.
\end{eqnarray}
However, for $\gamma \gg 1$, we have:
\begin{eqnarray}
\beta_c =  \pars{\frac{11}{320} + \frac{1}{1680 \log 4}}^{-\frac{1}{2}} \approx 5.36.
\end{eqnarray}