\appendix

%%%%%%%%%%%%%%%%%%%%%%%%%%%%%%%%%%%%%%%%%%%%%%%%%%%%%%%%%%%%%%%%%%%%%%%%%%%%%%%
\section{Nondimensionalization} \label{append:nondim}

The dimensional ionic fluxes are given by Nernst-Plank equation:
\begin{eqnarray}
\bj^*_\pm &=& 
-D^* \bnabla c^*_\pm + \bv^* c^*_\pm \mp \frac{z e^* D^*}{k_B^* T^*} c^*_\pm \bnabla \varphi^*.
\end{eqnarray}

The dimensional Stokes equation:
\begin{eqnarray}
-\bnabla p^* + \mu^* \bLaplacian \bv^* + \eps^* \Laplacian \varphi^* \bnabla \varphi^* &=& 0.
\end{eqnarray}


The electric potential $\varphi$ is normalized by the thermal voltage:
\begin{eqnarray}
\varphi^* &=& \frac{k_B^* T^*}{z e^*} = \frac{R^* T^*}{z F^*}
\end{eqnarray}

The velocity is normalized by:
\begin{eqnarray}
v^* &=& \frac{\eps^* (\varphi^*)^2}{a^* \mu^*}
\end{eqnarray}

The fluxes $\bj^*_\pm$ are normalized by:
\begin{eqnarray}
j^* = \frac{D^* c^*}{a^*}
\end{eqnarray}

The nondimensional Nernst-Plank equation is written as:
\begin{eqnarray}
\bj_\pm &=& 
-\bnabla c_\pm + \alpha \bv c_\pm \mp c_\pm \bnabla \varphi, \\
\alpha &=&
\frac{v^* a^*}{D^*} = \frac{\eps^* (\varphi^*)^2}{\mu^* D^*}.
\end{eqnarray}

\section{Slip Condition} \label{append:slip}

\subsection{Electric Potential}
The leading-order electric potential in the Debye layer:
\begin{eqnarray}  
\deriv{^2 \varphi}{\rho^2} = -q = -\frac{c_+ - c_-}{2} = 
-C\frac{\exp\left[+(\varPhi - \varphi)\right] - \exp\left[-(\varPhi - \varphi)\right]}{2} = 
C \sinh(\varphi - \varPhi).
\end{eqnarray}
Denote $\psi = \varphi - \varPhi$, so that $\psi(0) = \cV - \varPhi = \zeta$ and 
$\psi(\rho\rightarrow\infty) = 0$:
\begin{eqnarray}  
\deriv{^2 \psi}{\rho^2} &=& C \sinh(\psi) = 2 C \sinh{\frac{\psi}{2}}\cosh{\frac{\psi}{2}} ,
\\
2\deriv{\psi}{\rho} \deriv{^2 \psi}{\rho^2} &=& 
4 C \sinh{\frac{\psi}{2}}\cosh{\frac{\psi}{2}} \deriv{\psi}{\rho} ,
\\
\deriv{}{\rho} \pars{\deriv{\psi}{\rho}}^2 &=& 
\deriv{}{\rho} \pars{2 \sqrt{C} \sinh{\frac{\psi}{2}}}^2 .
\end{eqnarray}

Integrating with respect to boundary condtions yields:
\begin{eqnarray}  
\deriv{\psi}{\rho} &=& -2\sqrt{C}\sinh\frac{\psi}{2}, \\  
\pars{4\sinh\frac{\psi}{4}\cosh\frac{\psi}{4}}^{-1}\deriv{\psi}{\rho} &=& -\sqrt{C}, \\  
\frac{\pars{\tanh\frac{\psi}{4}}^{-1}}{4\cosh^2\frac{\psi}{4}}
\deriv{\psi}{\rho} &=& -\sqrt{C}, \\  
\deriv{}{\rho}\pars{\log\tanh\frac{\psi}{4}} &=& \deriv{}{\rho}\pars{-\rho \sqrt{C}} .
\end{eqnarray}

Integrating with respect to boundary condtions yields:
\begin{eqnarray}
\log\tanh\frac{\psi}{4} - \log\tanh\frac{\zeta}{4} &=& -\rho \sqrt{C}, \\
\psi = \varphi - \varPhi &=& 
4 \tanh^{-1}\pars{\tanh{\frac{\zeta}{4}} \cdot \exp\pars{-\rho \sqrt{C}}}.
\end{eqnarray}

\subsection{Radial Momentum}
Leading-order $O(\delta^{-3})$ radial momentum balance in the Debye layer yields:
\begin{eqnarray} 
- \deriv{p}{\rho} + \deriv{\varphi}{\rho} \deriv{^2\varphi}{\rho^2} &=& 0.
\end{eqnarray}
One integration yields:
\begin{eqnarray} 
p &=& \frac{1}{2}\pars{\deriv{\varphi}{\rho}}^2 = \frac{1}{2}\pars{\deriv{\psi}{\rho}}^2.
\end{eqnarray}

\subsection{Tangential Momentum}
Leading-order $O(\delta^{-2})$ tangential momentum balance in the Debye layer yields:
\begin{eqnarray} 
0 &=& \deriv{^2 v_\theta}{\rho^2} - \deriv{p}{\theta} 
 + \deriv{\varphi}{\theta} \deriv{^2\varphi}{\rho^2} ,
\\
\deriv{^2 v_\theta}{\rho^2} &=& \frac{1}{2}\deriv{}{\theta}\pars{\deriv{\psi}{\rho}}^2
- \deriv{(\psi + \varPhi)}{\theta} \deriv{^2\psi}{\rho^2}  
\\
 &=& \deriv{}{\theta}\pars{2C \sinh^2\frac{\psi}{2}}
- \deriv{\psi}{\theta} C \sinh\psi
- \deriv{\varPhi}{\theta} \deriv{^2\psi}{\rho^2}  
\\
 &=& \deriv{}{\theta}\pars{C \pars{\cosh\psi - 1}}
- C \sinh\psi \deriv{\psi}{\theta}
- \deriv{\varPhi}{\theta} \deriv{^2\psi}{\rho^2}  
\\
 &=& \deriv{C}{\theta} \pars{\cosh\psi - 1}
- \deriv{\varPhi}{\theta} \deriv{^2\psi}{\rho^2}  
 = \deriv{C}{\theta} \pars{2 \sinh^2 \frac{\psi}{2}}
- \deriv{\varPhi}{\theta} \deriv{^2\psi}{\rho^2}  
\\
 &=& -\deriv{C}{\theta} \frac{2}{\sqrt{C}} \pars{\half \sinh \frac{\psi}{2} \deriv{\psi}{\rho}}
- \deriv{\varPhi}{\theta} \deriv{^2\psi}{\rho^2}  
 = -\deriv{C}{\theta} \frac{2}{\sqrt{C}} \deriv{}{\rho}\pars{\cosh \frac{\psi}{2}}
- \deriv{\varPhi}{\theta} \deriv{^2\psi}{\rho^2}  .
\end{eqnarray}
By integrating from specific $\rho$ to $\rho \rightarrow \infty$ (where $\psi = 0$):
\begin{eqnarray}
\deriv{v_\theta}{\rho} &=& -\deriv{C}{\theta} \frac{2}{\sqrt{C}} 
\pars{\cosh \frac{\psi}{2} - 1} - \deriv{\varPhi}{\theta} \deriv{\psi}{\rho}   
= -\deriv{C}{\theta} \frac{4}{\sqrt{C}} \sinh^2 \frac{\psi}{4} 
  - \deriv{\varPhi}{\theta} \deriv{\psi}{\rho}   .
\end{eqnarray}
Note that:
\begin{eqnarray}
\deriv{\psi}{\rho} &=& -2\sqrt{C}\sinh\frac{\psi}{2} = 
                       -4\sqrt{C}\sinh\frac{\psi}{4}\cosh\frac{\psi}{4},
\\
-4\sinh\frac{\psi}{4} &=& \frac{1}{\sqrt{C} \cosh\frac{\psi}{4}} \deriv{\psi}{\rho}.
\end{eqnarray}
Thus:
\begin{eqnarray}
\deriv{v_\theta}{\rho} &=& 
  \pars{\deriv{C}{\theta} \frac{1}{C}} \cdot
  \pars{\frac{\sinh\frac{\psi}{4}}{\cosh\frac{\psi}{4}} \deriv{\psi}{\rho}}
  - \deriv{\varPhi}{\theta} \deriv{\psi}{\rho}
=  4\deriv{}{\rho} \pars{\log\cosh\frac{\psi}{4}} \deriv{}{\theta} \log C
  - \deriv{\psi}{\rho} \deriv{\varPhi}{\theta},
\\
\deriv{v_\theta}{\rho} &=& 
\deriv{}{\rho} \pars{ 4\log\cosh\frac{\psi}{4} \deriv{}{\theta} \log C
  - \psi \deriv{\varPhi}{\theta} }.
\end{eqnarray}

By integrating from $\rho = 0$ to $\rho \rightarrow \infty$, 
we have Dukhin-Derjaguin slip formula for tangential velocity component $V_\theta$ 
(where the radial component is $V_r = 0$) for $\zeta = \cV - \varPhi$:
\begin{eqnarray} 
V_\theta = \zeta \cdot \deriv{\varPhi}{\theta} -
      4\log\cosh \frac{\zeta}{4} \cdot \deriv{}{\theta} \log C 
= \zeta \cdot \deriv{\varPhi}{\theta} + 
      2\log\pars{1 - \tanh^2 \frac{\zeta}{4}} \cdot \deriv{}{\theta} \log C .
\end{eqnarray}

%%%%%%%%%%%%%%%%%%%%%%%%%%%%%%%%%%%%%%%%%%%%%%%%%%%%%%%%%%%%%%%%%%%%%%%%%%%%%%%
\section{Spherical Coordinates} \label{append:spherical}

The following spherical coordinate system is used:
\begin{eqnarray}
x &=& r \sin \theta \cos \phi \\
y &=& r \sin \theta \sin \phi \\
z &=& r \cos \theta
\end{eqnarray}

Assume full symmetry around $z$ axis -- there is no dependence on $\phi$:
\begin{eqnarray}
\bnabla f &=& \frac{\partial f}{\partial r} \mathbf{\hat{r}} +
\frac{1}{r} \frac{\partial f}{\partial \theta} \mathbf{\hat{\theta}} +
\frac{1}{r \sin\theta} \frac{\partial f}{\partial \phi} \mathbf{\hat{\phi}}
\\
\bnabla \mathbf{\cdot} \mathbf{F} &=& 
\frac{1}{r^2}\frac{\partial}{\partial r} \left( r^2 \cdot F_r\right)
  + \frac{1}{r \sin\theta} \frac{\partial}{\partial \theta} \left( \sin\theta \cdot F_\theta\right)
  + \frac{1}{r \sin\theta} \frac{\partial F_\phi}{\partial \phi}
\end{eqnarray}

\subsection{Scalar Operators}
Scalar Laplacian derivation:
\begin{eqnarray}
\Laplacian f = \bnabla \mathbf{\cdot} \bnabla f = \frac{1}{r^2}\frac{\partial}{\partial r}
\left( r^2 \frac{\partial f}{\partial r} \right) +
\frac{1}{r^2 \sin\theta} \frac{\partial}{\partial \theta} \left( \sin\theta \cdot \frac{\partial f}{\partial \theta}\right)
+ \frac{1}{r^2 \sin^2\theta} \frac{\partial^2 f}{\partial \phi^2}
\end{eqnarray}
In conservative form:
\begin{eqnarray}
r^2 \sin\theta \cdot \bLaplacian f = \frac{\partial}{\partial r} \left( r^2 \sin \theta \frac{\partial f}{\partial r} \right) +
\frac{\partial}{\partial \theta} \left( \sin\theta \cdot \frac{\partial f}{\partial \theta}\right) +
\frac{\partial}{\partial \phi} \left( \frac{1}{\sin\theta} \cdot \frac{\partial f}{\partial \phi}\right)
\end{eqnarray}

The unit vectors in spherical coordinate system are:
\begin{eqnarray}
 \br &=& \mat{c}{r\sin\theta \cos\phi \\ r\sin\theta \sin\phi \\ r\cos\theta} \\
 \brhat &=& \deriv{\br}{r}
 = \mat{c}{\sin\theta \cos\phi \\ \sin\theta \sin\phi \\ \cos\theta} \\
 \bthetahat &=& \frac{1}{r} \deriv{\br}{\theta}
 = \mat{c}{\cos\theta \cos\phi \\ \cos\theta \sin\phi \\ -\sin\theta} \\
 \bphihat &=& \frac{1}{r \sin \theta} \deriv{\br}{\phi}
 = \mat{c}{-\sin\phi \\ \cos\phi \\ 0}
\end{eqnarray}
Note that:
\begin{eqnarray}
  \deriv{\brhat}{r} = \deriv{\bthetahat}{r} = \deriv{\bphihat}{r} = \mathbf{0} \\
  \deriv{\brhat}{\theta} = \bthetahat \\
  \deriv{\bthetahat}{\theta} = -\brhat \\
  \deriv{\bphihat}{\theta} = \mathbf{0} \\
  \deriv{\brhat}{\phi} = \sin \theta \cdot \bphihat \\
  \deriv{\bthetahat}{\phi} = \cos \theta \cdot \bphihat \\
  \deriv{\bphihat}{\phi} = -\sin \theta \cdot \brhat -\cos \theta \cdot \bthetahat
\end{eqnarray}

\subsection{Vector Operators}
Vector gradient operator can be written by:
\begin{eqnarray}
\bnabla\bV &=& \left\{ \brhat \deriv{}{r} +
\bthetahat \frac{1}{r} \deriv{}{\theta} +
\bphihat \frac{1}{r \sin\theta} \deriv{}{\phi} \right\}
\bV
\\
\bnabla\bV &=& \left\{ \brhat \deriv{}{r} +
\bthetahat \frac{1}{r} \deriv{}{\theta} +
\bphihat \frac{1}{r \sin\theta} \deriv{}{\phi} \right\}
\left( V_r \brhat + V_\theta \bthetahat + V_\phi \bphihat \right)
\\
\deriv{\bV}{r} &=&
\deriv{V_r}{r}\brhat + \deriv{V_\theta}{r}\bthetahat + \deriv{V_\phi}{r}\bphihat
\\
\deriv{\bV}{\theta} &=&
\deriv{V_r}{\theta}\brhat + \deriv{V_\theta}{\theta}\bthetahat + \deriv{V_\phi}{\theta}\bphihat
+ V_r\bthetahat - V_\theta\brhat
\\
\deriv{\bV}{\phi} &=&
\deriv{V_r}{\phi}\brhat + \deriv{V_\theta}{\phi}\bthetahat + \deriv{V_\phi}{\phi}\bphihat
+ V_r \sin\theta\bphihat + V_\theta \cos\theta\bphihat -
V_\phi \left(\brhat \sin\theta + \bthetahat \cos\theta \right)
\\
\bnabla\bV &=&
\mat{ccc}{1&0&0 \\ 0&\frac{1}{r}&0 \\ 0&0&\frac{1}{r \sin\theta}}
\mat{ccc}{
\deriv{V_r}{r} & \deriv{V_\theta}{r} & \deriv{V_\phi}{r} \\
\deriv{V_r}{\theta} - V_\theta &
\deriv{V_\theta}{\theta} + V_r&
\deriv{V_\phi}{\theta} \\
\deriv{V_r}{\phi} - V_\phi \sin\theta &
\deriv{V_\theta}{\phi} - V_\phi \cos\theta  &
\deriv{V_\phi}{\phi} + V_r \sin\theta + V_\theta \cos\theta
}
\end{eqnarray}
Vector Laplacian components can be written as:
\begin{eqnarray}
\Laplacian \bV = \bnabla \cdot \bnabla \bV &=& \frac{1}{r^2}\deriv{}{r}\left(r^2 \deriv{\bV}{r}\right) +
\frac{1}{r^2\sin\theta}\deriv{}{\theta}\left(\sin\theta\deriv{\bV}{\theta}\right)+
\frac{1}{r^2 \sin^2\theta}\deriv{}{\phi}\left(\deriv{\bV}{\phi}\right)
\end{eqnarray}
$\brhat$ component:
\begin{eqnarray}
\Laplacian (V_r\brhat) &=&
\frac{1}{r^2}\deriv{}{r}\left(r^2 \deriv{V_r}{r}\right)\brhat
\\ &+&
\frac{1}{r^2\sin\theta}\deriv{}{\theta}\left(
\sin\theta\left(\deriv{V_r}{\theta}\brhat + V_r\bthetahat \right)\right)
\\ &+&
\frac{1}{r^2 \sin^2\theta}\deriv{}{\phi}\left(
\deriv{V_r}{\phi}\brhat + V_r \sin\theta\bphihat
\right) \\
\Laplacian (V_r\brhat) &=&
\left(
\frac{2}{r}\deriv{V_r}{r}+\deriv{^2V_r}{r^2}\right)\brhat
\\ &+&
\frac{1}{r^2\sin\theta}\left(
\cos\theta\left(\deriv{V_r}{\theta}\brhat + V_r\bthetahat \right) +
\sin\theta\left(\deriv{^2V_r}{\theta^2}\brhat + \deriv{V_r}{\theta}\bthetahat + \deriv{V_r}{\theta}\bthetahat - V_r\brhat\right)
\right)
\\ &+&
\frac{1}{r^2 \sin^2\theta}\left(
\deriv{^2V_r}{\phi^2}\brhat + \deriv{V_r}{\phi} \sin\theta\bphihat +
\deriv{V_r}{\phi}\bphihat\sin\theta - V_r \sin\theta
\left(\brhat \sin\theta + \bthetahat \cos\theta \right)
\right)
\\
\Laplacian (V_r\brhat) &=&
\left(
\frac{2}{r}\deriv{V_r}{r}+\deriv{^2V_r}{r^2}-\frac{2V_r}{r^2}
+ \frac{1}{r^2}\deriv{^2V_r}{\theta^2}
+ \frac{\cot\theta}{r^2} \deriv{V_r}{\theta}
+ \frac{1}{r^2 \sin^2\theta} \deriv{^2V_r}{\phi^2}
\right)\brhat
\\ &+& \frac{2}{r^2} \deriv{V_r}{\theta} \bthetahat + \frac{2}{r^2 \sin\theta}\deriv{V_r}{\phi} \bphihat
\end{eqnarray}
$\bthetahat$ component:
\begin{eqnarray}
\Laplacian (V_\theta\bthetahat) &=&
\frac{1}{r^2}\deriv{}{r}\left(r^2 \deriv{V_\theta}{r}\right)\bthetahat
\\ &+&
\frac{1}{r^2\sin\theta}\deriv{}{\theta}\left(
\sin\theta\left(\deriv{V_\theta}{\theta}\bthetahat - V_\theta\brhat \right)\right)
\\ &+&
\frac{1}{r^2 \sin^2\theta}\deriv{}{\phi}\left(
\deriv{V_\theta}{\phi}\bthetahat + V_\theta \cos\theta\bphihat
\right)
\\
\Laplacian (V_\theta\bthetahat) &=&
\left(\frac{2}{r}\deriv{V_\theta}{r} + \deriv{^2V_\theta}{r^2}\right)\bthetahat
\\ &+&
\frac{1}{r^2\sin\theta}\left(
\cos\theta\left(\deriv{V_\theta}{\theta}\bthetahat - V_\theta\brhat \right) +
\sin\theta\left(\deriv{^2V_\theta}{\theta^2}\bthetahat - \deriv{V_\theta}{\theta}\brhat
-\deriv{V_\theta}{\theta}\brhat - V_\theta\bthetahat\right)
\right)
\\ &+&
\frac{1}{r^2 \sin^2\theta}\left(
\deriv{^2V_\theta}{\phi^2}\bthetahat + \deriv{V_\theta}{\phi} \cos\theta\bphihat +
\deriv{V_\theta}{\phi}\cos\theta\bphihat
- V_\theta \cos\theta \left(\brhat \sin\theta + \bthetahat \cos\theta \right)
\right)
\\
\Laplacian (V_\theta\bthetahat) &=&
\left(-\frac{2\cot\theta}{r^2}V_\theta - 2\deriv{V_\theta}{\theta}\right)\brhat +
\left(\frac{2}{r}\deriv{V_\theta}{r} + \deriv{^2V_\theta}{r^2} +
\frac{\cot\theta}{r^2} \deriv{V_\theta}{\theta} +
\frac{1}{r^2}\deriv{^2V_\theta}{\theta^2}\right)\bthetahat
\\
\\ &+&
\frac{1}{r^2 \sin^2\theta}\left(
\deriv{^2V_\theta}{\phi^2} - V_\theta \right)\bthetahat
+ \frac{2 \cot\theta}{r^2 \sin\theta} \deriv{V_\theta}{\phi} \bphihat
\end{eqnarray}
$\bphihat$ component:
\begin{eqnarray}
\Laplacian (V_\phi\bphihat) &=&
\frac{1}{r^2}\deriv{}{r}\left(r^2 \deriv{V_\phi}{r}\right)\bphihat +
\frac{1}{r^2\sin\theta}\deriv{}{\theta}\left(
\sin\theta \deriv{V_\phi}{\theta}\bphihat \right)
\\ &+&
\frac{1}{r^2 \sin^2\theta}\deriv{}{\phi}\left(
\deriv{V_\phi}{\phi}\bphihat
- V_\phi \left(\brhat \sin\theta + \bthetahat \cos\theta\right)
\right)
\\
\Laplacian (V_\phi\bphihat) &=&
\frac{1}{r^2}\deriv{}{r}\left(r^2 \deriv{V_\phi}{r}\right)\bphihat +
\frac{1}{r^2\sin\theta}\deriv{}{\theta}\left(
\sin\theta \deriv{V_\phi}{\theta}\bphihat \right)
\\ &+&
\frac{1}{r^2 \sin^2\theta}\deriv{}{\phi}\left(
\deriv{V_\phi}{\phi}\bphihat
- V_\phi \left(\brhat \sin\theta + \bthetahat \cos\theta\right)
\right)
\\
\Laplacian (V_\phi\bphihat) &=&
\left(\frac{2}{r} \deriv{V_\phi}{r} + \deriv{^2V_\phi}{r^2}\right)\bphihat
+
\frac{1}{r^2\sin\theta}\left(
\cos\theta \deriv{V_\phi}{\theta} +
\sin\theta \deriv{^2V_\phi}{\theta^2}
\right)\bphihat
\\ &+&
\frac{1}{r^2 \sin^2\theta}\left(
\deriv{^2V_\phi}{\phi^2}\bphihat
- 2 \deriv{V_\phi}{\phi} \left(\brhat \sin\theta + \bthetahat \cos\theta\right)
- V_\phi \bphihat
\right)
\end{eqnarray}
In summary:
\begin{eqnarray}
\Laplacian \bV &=& \left(\Laplacian V_r - \frac{2V_r}{r^2}\right)\brhat
+ \frac{2}{r^2}\deriv{V_r}{\theta} \bthetahat + \frac{2}{r^2 \sin\theta}\deriv{V_r}{\phi} \bphihat \\
&-&
\frac{2}{r^2 \sin\theta} \deriv{\left(V_\theta \sin\theta \right)}{\theta}\brhat
+ \left(\Laplacian V_\theta - \frac{V_\theta}{r^2 \sin^2\theta}\right) \bthetahat
+ \frac{2 \cot\theta}{r^2 \sin\theta} \deriv{V_\theta}{\phi} \bphihat \\
&-&
\frac{2}{r^2 \sin\theta}\deriv{V_\phi}{\phi}\brhat -
\frac{2 \cot\theta}{r^2 \sin\theta}\deriv{V_\phi}{\phi}\bthetahat
+ \left(\Laplacian V_\phi - \frac{V_\phi}{r^2 \sin^2\theta}\right)\bphihat
\end{eqnarray}

Since $V_\phi = 0$ and $\deriv{}{\phi}(\cdot) = 0$, we get:
\begin{eqnarray}
\Laplacian \bV &=& \left(\Laplacian V_r - \frac{2V_r}{r^2} - \frac{2}{r^2 \sin\theta} \deriv{\left(V_\theta \sin\theta \right)}{\theta}\right)\brhat
\\
&+&
\left(\Laplacian V_\theta - \frac{V_\theta}{r^2 \sin^2\theta} + \frac{2}{r^2}\deriv{V_r}{\theta}\right) \bthetahat
\\
\Laplacian \bV &=& \left(
\frac{1}{r^2}\deriv{}{r}\left( r^2 \deriv{V_r}{r} \right) + \frac{1}{r^2 \sin\theta} \deriv{}{\theta} \left( \sin\theta \cdot \deriv{V_r}{\theta}\right)
 - \frac{2V_r}{r^2} - \frac{2}{r^2 \sin\theta} \deriv{\left(V_\theta \sin\theta \right)}{\theta}\right)\brhat \\
&+& \left(
\frac{1}{r^2}\deriv{}{r}\left( r^2 \deriv{V_\theta}{r} \right) + \frac{1}{r^2 \sin\theta} \deriv{}{\theta} \left( \sin\theta \cdot \deriv{V_\theta}{\theta}\right)
 - \frac{V_\theta}{r^2 \sin^2\theta} + \frac{2}{r^2}\deriv{V_r}{\theta}\right) \bthetahat
\end{eqnarray}
