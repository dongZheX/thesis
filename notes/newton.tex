\section{Newton's method}
\subsection{Algorithm}
If we write the system as $f(\varPhi,C,\bV,P) = 0$, we may use Netwon's method to solve it, by linearizing $F$ around current state variables
and using the Hessian to compute the Newton's step.
\begin{eqnarray}
  f(x) &\approx & f(x_0) + H(x_0) (x - x0)
\end{eqnarray}
Thus, the Newton's step to solve $f(x) = 0$,
defined as $\Delta x = x - x_0$, is:
\begin{eqnarray}
  \Delta x &=& -H^{-1}(x_0) f(x_0)
\end{eqnarray}
Thus, we have to compute the Hessian $H$ of $f$ at $x_0$.

\subsection{Hessian derivation}
$f$ is a vector function describing Laplace's, advection and Stokes'
equations.
\begin{eqnarray}
  f &=& \left[\begin{array}{c}
  \bnabla \cdot (C \bnabla \varPhi) \\
  \Laplacian C - \alpha \bV \cdot \bnabla C \\
  \bLaplacian \bV - \bnabla P + \Laplacian \varPhi \bnabla \varPhi
  \end{array}\right]
  \\
  df & \approx & \left[\begin{array}{cccc}
  \bnabla \cdot (C \bnabla \phi) &+ \bnabla \cdot (c \bnabla \varPhi) & & \\
  &\Laplacian c - \alpha \bV \cdot \bnabla c &- \alpha \bv \cdot \bnabla C&\\
  \Laplacian \phi \bnabla \varPhi +
  \Laplacian \varPhi \bnabla \phi &
  & +\bLaplacian \bv &- \bnabla p
  \end{array}\right] \\
  df & \approx & H(\varPhi,C,\bV,P) \left[\begin{array}{c}
  \phi \\ c \\ \bv \\ p
  \end{array}\right]
\end{eqnarray}

\subsection{Boundary conditions}
\begin{eqnarray}
r = 1 &\Rightarrow&
\brc{rcl}{
  \varPhi & = & -\log C \\
  \partial_r {\varPhi} & = & \partial_r {\log C} \\
  \bV & = & \left\{\log (C/\gamma) \partial_\theta \varPhi +
  2 \log\left(1 - \tanh^2 \frac{\log (C/\gamma)}{4}\right)
  \partial_\theta \log C \right\} \bthetahat
}
\\
r \rightarrow \infty &\Rightarrow&
\brc{rcl}{
  \bnabla \varPhi &\rightarrow& -\beta \bxhat \\
  C &\rightarrow& 1 \\
  \bV &\rightarrow& -\mathcal{U} \bxhat
}
\end{eqnarray}

Apply linearization:
\begin{eqnarray}
r = 1 &\Rightarrow&
\brc{rcl}{
  0 & = & \phi + {c / C} \\
  \partial_r {\phi} & = &  \partial_r {(c/C)} \\
  \bv & = & \left\{c/C \partial_\theta \varPhi +
  \log (C/\gamma) \partial_\theta \phi \right\}\bthetahat
  \\ & &
  + 2 \log\left(1 - \tanh^2 \frac{\log (C/\gamma)}{4}\right)
  \partial_\theta (c/C) \cdot \bthetahat
  \\ & &
  - c/C \tanh \frac{\log (C/\gamma)}{4}
  \partial_\theta \log C \cdot \bthetahat
}
\\
r \rightarrow \infty &\Rightarrow&
\brc{rcl}{
  \partial_r \phi &\rightarrow& 0 \\
  c &\rightarrow& 0 \\
  \bv &\rightarrow& \bzero
}
\end{eqnarray}

\subsection{Slipping}
\begin{eqnarray}
  \xi &=& \log \frac{C}{\gamma} = -\varPhi - \log\gamma \\
  V_\theta & = & \xi \partial_\theta \varPhi
  + 2 \log\left(1 - \tanh^2 \frac{\xi}{4}\right) \partial_\theta \log C \\
  V_\theta & = & \xi \partial_\theta \varPhi
  - 2 \log\left(1 - \tanh^2 \frac{\xi}{4}\right) \partial_\theta \varPhi 
  \\ V_\theta & = & \left( \xi + 2 \log\left(\cosh^2 \frac{\xi}{4}\right) \right) \partial_\theta \varPhi
  \\ V_\theta & = & \left( \xi + 4 \log\left(\cosh \frac{\xi}{4}\right) \right) \partial_\theta \varPhi
  \\ V_\theta & = & 4 \log\left(e^\frac{\xi}{4} \frac{ e^\frac{\xi}{4} + e^{-\frac{\xi}{4}} }{2} \right) \partial_\theta \varPhi
  \\ V_\theta & = & 4 \log\left( \frac{ e^\frac{\xi}{2} + 1 }{2} \right) \partial_\theta \varPhi
\end{eqnarray}
For $\beta \ll 1$, we have $\xi \approx -\log\gamma$:
\begin{eqnarray}
  V_\theta & \approx & 4 \log\left( \frac{ \gamma^{-\frac{1}{2}} + 1 }{2} \right) \partial_\theta \varPhi \\
  V_\theta & \approx & 4 \log\left( \frac{ \gamma^{\frac{1}{4}} + \gamma^{-\frac{1}{4}} }{2 \gamma^{\frac{1}{4}}} \right) \partial_\theta \varPhi
\end{eqnarray}

