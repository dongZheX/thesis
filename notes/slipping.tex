\section{Slip velocity}
\subsection{Correction for Stokes solution}
Denote stream function $\Psi$ as:
\begin{eqnarray}
  V_r &=& \frac{1}{r^2 \sin\theta}\deriv{\Psi}{\theta} \\
  V_\theta &=& -\frac{1}{r \sin\theta}\deriv{\Psi}{r}
\end{eqnarray}
We use the same solution form for $\Psi$ as before:
\begin{eqnarray}
  \Psi &=& (A r^2 + B r + C r^{-1})\sin^2\theta\\
  V_r(\infty, \theta) = 2A \cos\theta &=& 0 \\
  V_\theta(\infty, \theta) = -2A \sin\theta &=& 0 \\
  V_r(1, \theta) = (A+B+C)\cdot 2\cos\theta &=& 0 \\
  V_\theta(1, \theta) = -(2A+B-C)\cdot \sin\theta &=& -W\sin\theta
\end{eqnarray}
Thus:
\begin{eqnarray}
  \brc{ccc}{A &=& 0 \\ B &=& W/2 \\ C &=& -W/2}
\end{eqnarray}
So the stream function for slipping velocity $V_\theta = -W\sin\theta$
is $\Psi = \frac{W}{2}(r - r^{-1})\sin^2\theta$.

Substituting into Stokes equation we get that:
\begin{eqnarray}
  \bnabla P = \bLaplacian \bV &=& -\frac{2 W \cos\theta}{r^3} \brhat
  -\frac{W \sin \theta}{r^3} \bthetahat \\
  P &=& P_\infty + \frac{W \cos\theta}{r^2}
\end{eqnarray}

The force acting on the sphere area element in $\bxhat$ direction is:
\begin{eqnarray}
  F &=& \frac{3W \cos(2\theta)}{2}
\end{eqnarray}
Integrated over the whole sphere, the total force is:
\begin{eqnarray}
  F_T &=& -4\pi W
\end{eqnarray}
Thus, in steady state, we have $W = \frac{3}{2} \cl{U}$,
in order to balance Stokes drag.
\subsection{Steady state for small $\beta$}
\begin{eqnarray}
  \varPhi &=& \beta \brcs{\frac{1}{4r^2} - r} \cos\theta \\
  C &=&  1 + \beta\frac{3}{4r^2} \cos\theta \\
  \Psi &=& \frac{\cl{U} \sin^2\theta}{2} \brcs{r^2 - \frac{a^3}{r}} \\
  V_r &=& \cl{U} \brcs{1 - \frac{a^3}{r^3}} \cos\theta \\
  V_\theta &=& -\cl{U} \brcs{1 + \frac{a^3}{2r^3}} \sin\theta \\
  P &=& P_\infty
\end{eqnarray}

Since we have a potential flow, we may write that:
\begin{eqnarray}
  \bV &=& \cl{U}\bnabla \left\{\brcs{r + \frac{a^3}{2r^2}} \cos\theta\right\}
\end{eqnarray}

Note that though the total force is zero,
the force acting on each sphere area element is given by:
\begin{eqnarray}
  \bF &=& \frac{3\mu\cl{U}}{a}
  \brcs{ 2\cos\theta \cdot \brhat + \sin\theta \cdot \bthetahat} \\
  \bF \cdot \bxhat &=& \frac{3\mu\cl{U}}{a} \brcs{ 3\cos^2\theta - 1}
\end{eqnarray}
For small $\beta$, we get using Dukhin-�Derjaguin for slip velocity:
\begin{eqnarray}
V_\theta & \approx & -\beta \cdot 3
\log\brcs{\frac{\gamma^{1/4} + \gamma^{-1/4}}{2\gamma^{1/4}}} \sin\theta \\
\cl{U}_\infty = \frac{2}{3} W &\approx& \beta \cdot 2
\log\brcs{\frac{\gamma^{1/4} + \gamma^{-1/4}}{2\gamma^{1/4}}} \\
\cl{U}_\infty &\approx& \beta \cdot 2
\log\brcs{\frac{1 + \gamma^{-1/2}}{2}}
\end{eqnarray}

Note that Dukhin-Derjaguin slip formula may be rewritten as:
\begin{eqnarray}
  \xi &=& \log \frac{C}{\gamma} = -\varPhi - \log\gamma \\
  V_\theta & = & \xi \partial_\theta \varPhi
  + 2 \log\left(1 - \tanh^2 \frac{\xi}{4}\right) \partial_\theta \log C \\
  V_\theta & = & \xi \partial_\theta \varPhi
  - 2 \log\left(1 - \tanh^2 \frac{\xi}{4}\right) \partial_\theta \varPhi
  \\ V_\theta & = & \left( \xi + 2 \log\left(\cosh^2 \frac{\xi}{4}\right) \right) \partial_\theta \varPhi
  \\ V_\theta & = & \left( \xi + 4 \log\left(\cosh \frac{\xi}{4}\right) \right) \partial_\theta \varPhi
  \\ V_\theta & = & 4 \log\left(e^\frac{\xi}{4} \frac{ e^\frac{\xi}{4} + e^{-\frac{\xi}{4}} }{2} \right) \partial_\theta \varPhi
  \\ V_\theta & = & 4 \log\left( \frac{ e^\frac{\xi}{2} + 1 }{2} \right) \partial_\theta \varPhi
\end{eqnarray}
For $\beta \ll 1$, we have $\xi \approx -\log\gamma$:
\begin{eqnarray}
  V_\theta & \approx & 4 \log\left( \frac{ \gamma^{-\frac{1}{2}} + 1 }{2} \right) \partial_\theta \varPhi \\
  V_\theta & \approx & 4 \log\left( \frac{ \gamma^{\frac{1}{4}} + \gamma^{-\frac{1}{4}} }{2 \gamma^{\frac{1}{4}}} \right) \partial_\theta \varPhi \\
  V_\theta & \approx & 4 \log\left( \frac{ \gamma^{\frac{1}{4}} + \gamma^{-\frac{1}{4}} }{2 \gamma^{\frac{1}{4}}} \right) \cdot \frac{3}{4} \beta \sin\theta \\
  V_\theta & \approx & 3 \log\left( \frac{ \gamma^{\frac{1}{4}} + \gamma^{-\frac{1}{4}} }{2 \gamma^{\frac{1}{4}}} \right) \beta \sin\theta \\
  \cal{U} &=& 2 \log\left( \frac{ \gamma^{\frac{1}{4}} + \gamma^{-\frac{1}{4}} }{2 \gamma^{\frac{1}{4}}} \right) \beta
\end{eqnarray}