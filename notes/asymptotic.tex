\section{Asymptotic expansion}
Coupled equations system:
\begin{eqnarray}
 \brc{lll}{
  \bnabla \cdot \brcs{C \bnabla \varPhi} &=& 0 \\
  \Laplacian C - \bV \cdot \bnabla C &=& 0 \\
  \bLaplacian V - \bnabla P + \Laplacian \varPhi \bnabla\varPhi &=& 0 }
  \\
 \brc{rcl}{
  \varPhi &=& \beta\varPhi_1 + \beta^2\varPhi_2 \\
  C &=& 1 + \beta C_1 + \beta^2 C_2 \\
  \bV &=& \beta \bV_1 + \beta^2 \bV_2 \\
  P &=& \beta P_1 + \beta^2 P_2}
\end{eqnarray}
Substitute:
\begin{eqnarray}
 \brc{lll}{
  \bnabla \cdot \brcs{(1 + \beta C_1 + \beta^2 C_2) \bnabla (\beta\varPhi_1 + \beta^2\varPhi_2)} &=& 0 \\
  \Laplacian (\beta C_1 + \beta^2 C_2) - (\beta \bV_1 + \beta^2 \bV_2) \cdot \bnabla (\beta C_1 + \beta^2 C_2) &=& 0 \\
  \bLaplacian (\beta \bV_1 + \beta^2 \bV_2) - \bnabla (\beta P_1 + \beta^2 P_2) + \Laplacian (\beta\varPhi_1 + \beta^2\varPhi_2) \bnabla(\beta\varPhi_1 + \beta^2\varPhi_2) &=& 0 }
\end{eqnarray}
Discard terms above $O(\beta^2)$:
\begin{eqnarray}
 \brc{lll}{
  \bnabla \cdot \brcs{(1 + \beta C_1) \bnabla (\beta \varPhi_1) + \bnabla(\beta^2\varPhi_2)} &=& 0 \\
  \Laplacian (\beta C_1 + \beta^2 C_2) - \beta \bV_1 \cdot \bnabla (\beta C_1) &=& 0 \\
  \bLaplacian (\beta \bV_1 + \beta^2 \bV_2) - \bnabla (\beta P_1 + \beta^2 P_2) + \Laplacian (\beta\varPhi_1) \bnabla(\beta\varPhi_1) &=& 0 }
\end{eqnarray}
Extract $O(\beta)$ equation:
\begin{eqnarray}
 \brc{lll}{
  \bnabla \cdot \brcs{\bnabla (\beta\varPhi_1} &=& 0 \\
  \Laplacian (\beta C_1) &=& 0 \\
  \bLaplacian (\beta \bV_1) - \bnabla (\beta P_1) &=& 0 }
\end{eqnarray}
Extract $O(\beta^2)$ equation:
\begin{eqnarray}
 \brc{lll}{
  \Laplacian (\beta^2 \varPhi_2) &=& - \bnabla \cdot \brcs{(\beta + \beta^2 C_1) \bnabla \varPhi_1}
  \\
  \Laplacian (\beta^2 C_2) &=& \beta \bV_1 \cdot \bnabla (\beta C_1) -\Laplacian (\beta C_1)
  \\
  \bLaplacian (\beta^2 \bV_2) - \bnabla (\beta^2 P_2) &=& - \bLaplacian (\beta \bV_1) + \bnabla (\beta P_1) - \Laplacian (\beta\varPhi_1) \bnabla(\beta\varPhi_1)}
\end{eqnarray}
Divide equations by $\beta$ and simplify:
\begin{eqnarray}
 \brc{lll}{
  \beta \Laplacian \varPhi_2 &=& - \bnabla \cdot \brcs{(1 + \beta C_1) \bnabla \varPhi_1}
  \\
  \beta \Laplacian C_2 &=& \beta \bV_1 \cdot \bnabla C_1 -\Laplacian C_1
  \\
  \beta\brcs{\bLaplacian \bV_2 - \bnabla P_2} &=& - \bLaplacian \bV_1 + \bnabla P_1 - \beta \Laplacian \varPhi_1 \bnabla \varPhi_1}
\end{eqnarray}
Since $\Laplacian\varPhi_1 = \Laplacian C_1 = 0$ and $\bLaplacian \bV_1 = \bnabla P_1$, we rewrite:
\begin{eqnarray}
 \brc{lll}{
  \beta \Laplacian \varPhi_2 &=& - \bnabla \cdot \brcs{(\beta C_1) \bnabla \varPhi_1}
  \\
  \beta \Laplacian C_2 &=& \beta \bV_1 \cdot \bnabla C_1
  \\
  \beta\brcs{\bLaplacian \bV_2 - \bnabla P_2} &=& 0
 }
\end{eqnarray}
Thus the equations for $O(\beta^2)$ have the same form as for $O(\beta)$, but with non-zero RHS in Laplace and Advection equations:
\begin{eqnarray}
 \brc{lll}{
  \Laplacian \varPhi_2 &=& -\bnabla C_1 \cdot \bnabla \varPhi_1
  \\
  \Laplacian C_2 &=& \bV_1 \cdot \bnabla C_1
  \\
  \bLaplacian \bV_2 - \bnabla P_2 &=& 0
 }
\end{eqnarray}
\begin{eqnarray}
  \varPhi_1 &=& \brcs{\frac{1}{4r^2} - r} \cos\theta \\
  C_1 &=&  \frac{3}{4r^2} \cos\theta \\
  V_r &=& 2 \log\brcs{\frac{1 + \gamma^{-1/2}}{2}} \brcs{1 - \frac{a^3}{r^3}} \cos\theta \\
  V_\theta &=& -2 \log\brcs{\frac{1 + \gamma^{-1/2}}{2}} \brcs{1 + \frac{a^3}{2r^3}} \sin\theta 
\end{eqnarray}

Boundary conditions:
\begin{eqnarray}
  C &=& \exp(-\varPhi) \\
  \deriv{C}{r} &=& C\deriv{\varPhi}{r}
\end{eqnarray}
Linearized boundary conditions (since $\exp(x) \approx 1 + x + x^2/2$):
\begin{eqnarray}
  1 + \beta C_1 + \beta^2 C_2  &=& \exp(-\beta \varPhi_1 - \beta^2 \varPhi_2) \\
  &=& 1 -\beta \varPhi_1 - \beta^2 \varPhi_2 + (\beta \varPhi_1 + \beta^2 \varPhi_2)^2/2 + o(\beta^2) \\
  &=& 1 -\beta \varPhi_1 - \beta^2 \varPhi_2 + \beta^2 \varPhi_1^2/2 + o(\beta^2) \\
  C_2 &=& -\varPhi_2 + \varPhi_1^2/2 \\
  \beta \deriv{C_1}{r} + \beta^2 \deriv{C_2}{r} &=& 
   \left(1 + \beta C_1 + \beta^2 C_2\right)\left(\beta \deriv{\varPhi_1}{r} + \beta^2 \deriv{\varPhi_2}{r}\right)   
   \\ \deriv{C_2}{r} &=& \deriv{\varPhi_2}{r} + C_1 \deriv{\varPhi_1}{r}
\end{eqnarray}

Slipping velocity is:
\begin{eqnarray}
V_\theta(\varPhi) &=& 4\log\left(\frac{e^\frac{-\varPhi-\log\gamma}{2}+1}{2}\right)\deriv{\varPhi}{\theta} 
= g(\varPhi) \deriv{\varPhi}{\theta} 
\\
V_\theta(\varPhi) &=& \beta \cdot g(0) \deriv{\varPhi_1}{\theta} +
\beta^2 \left(g(0) \deriv{\varPhi_2}{\theta} + g'(0) \varPhi_1 \deriv{\varPhi_1}{\theta}\right)
\\
\bV_2 &=& (3\sin^2\theta - 2)\left(r^{-1} - r^{-3}\right) \brhat + \sin2\theta\frac{r^-1+r^-3}{2}\bthetahat
\end{eqnarray}

Stress tensor is:
\begin{eqnarray}
  \mathbb{S} &=& -P \mathbb{I} + \bnabla \bV + (\bnabla \bV)^\dagger +
  \bnabla \varPhi \bnabla \varPhi - \frac{1}{2} (\bnabla \varPhi \cdot \bnabla \varPhi) \mathbb{I}
  \\
  \mathbb{S}_1 &=& -P_1 \mathbb{I} + \bnabla \bV_1 + (\bnabla \bV_1)^\dagger
  \\
  \mathbb{S}_2 &=& -P_2 \mathbb{I} + \bnabla \bV_2 + (\bnabla \bV_2)^\dagger + \bnabla \varPhi_1 \bnabla \varPhi_1 - \frac{1}{2} (\bnabla \varPhi_1 \cdot \bnabla \varPhi_1) \mathbb{I}
\end{eqnarray} 

\subsection{Asymptotic expansion in $\beta$}
For $a=0$, we can write:
\begin{eqnarray}
\mathcal{U} = 2\log\left(\frac{1+\gamma^{-1/2}}{2}\right)\beta 
       + \left(\frac{31}{320\left(\gamma^{1/2}+1\right)} 
       - \frac{9}{320\left(\gamma^{1/2}+1\right)^2} 
       - \frac{11}{160}\log\left(\frac{1+\gamma^{-1/2}}{2}\right)\right)\beta^3
\end{eqnarray} 
