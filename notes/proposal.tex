\documentclass[11pt]{article}

\usepackage{amsmath} % Required for \eqref
\usepackage{amssymb} % Required for \mathbb
\usepackage{units}   % Required for \nicefrac
\usepackage{float}   % Required for algorithm and floating environments
\usepackage{graphicx}% Required for figures and imagesf
\usepackage{amsfonts}
\usepackage{mathrsfs}
\usepackage[colorlinks]{hyperref}
\usepackage{framed}


%-------------------------------------------
\newtheorem{theorem}{Theorem}
\newtheorem{lemma}[theorem]{Lemma}
\newtheorem{acknowledgment}[theorem]{Acknowledgment}
\newtheorem{proposition}[theorem]{Proposition}
\newtheorem{corollary}[theorem]{Corollary}
%-------------------------------------------
\floatstyle{ruled}
\newfloat{program}{thp}{lop}
%-------------------------------------------
\newcommand{\sign}{\ensuremath{\mathrm{sign}}}
\newcommand{\diag}{\ensuremath{\mathrm{diag}}}
\newcommand{\trace}[1]{\ensuremath{\mathrm{trace}\left( #1 \right)}}
\newcommand{\norm}[1]{\ensuremath{\left\|#1\right\|_2^2}}
\newcommand{\func}[2]{\ensuremath{\mathrm{#1}\left( #2 \right)}}
\newcommand\eps \epsilon
\newcommand{\R}{\ensuremath{\mathbb{R}}}
\newcommand{\N}{\ensuremath{\mathbb{N}}}
\newcommand{\real}{\ensuremath{\mathbb{R}}}
\newcommand{\cl}[1]{\ensuremath{\mathcal{#1}}}
\newcommand{\suppsize}[1]{\ensuremath{|\mathcal{#1}|}}
\newcommand{\vect}[1]{\ensuremath{\mathbf{#1}}}
\newcommand{\matr}[1]{\ensuremath{\mathbf{#1}}}
\newcommand{\deriv}[2]{\frac{\partial #1}{\partial #2}}
\newcommand{\arr}[2]{\begin{array}{#1}#2\end{array}}
\newcommand{\mat}[2]{\left(\begin{array}{#1}#2\end{array}\right)}
\newcommand{\brc}[2]{\left\{\begin{array}{#1}#2\end{array}\right.}
\newcommand{\pars}[1]{\left(#1\right)}
\newcommand{\brcs}[1]{\left\{#1\right\}}
\newcommand{\half}{\frac{1}{2}}

% Bold symbols and operators
\newcommand\Laplacian{\nabla^2}
\newcommand\bnabla{\boldsymbol{\nabla}}
\newcommand\bLaplacian{\boldsymbol{\nabla}^2}
\newcommand\bcdot{\boldsymbol{\cdot}}
\newcommand\bU{\mathscr{\boldsymbol{U}}}
\newcommand\bv{\boldsymbol{v}}
\newcommand\bV{\boldsymbol{V}}
\newcommand\bE{\boldsymbol{E}}
\newcommand\be{\boldsymbol{\hat{e}}}
\newcommand\bn{\boldsymbol{\hat{n}}}
\newcommand\bk{\boldsymbol{\hat{k}}}
\newcommand\bj{\boldsymbol{j}}
\newcommand\bi{\boldsymbol{i}}
\newcommand\bA{\boldsymbol{A}}
\newcommand\bF{\boldsymbol{F}}
\newcommand\bI{\boldsymbol{I}}
\newcommand\bJ{\boldsymbol{J}}
\newcommand\bx{\boldsymbol{x}}
\newcommand\by{\boldsymbol{y}}
\newcommand\bz{\boldsymbol{z}}
\newcommand\br{\boldsymbol{r}}
\newcommand\bc{\boldsymbol{c}}
\newcommand\bxhat{\hat{\bx}}
\newcommand\byhat{\hat{\by}}
\newcommand\bzhat{\hat{\bz}}
\newcommand\brhat{\hat{\br}}
\newcommand\bnhat{\hat{\boldsymbol{n}}}
\newcommand\btheta{\boldsymbol{\theta}}
\newcommand\bthetahat{\hat{\btheta}}
\newcommand\bphi{\boldsymbol{\phi}}
\newcommand\bphihat{\hat{\bphi}}
\newcommand\bzero{\boldsymbol{0}}
\newcommand\bomega{\boldsymbol{\omega}}
\newcommand\bpsi{\boldsymbol{\psi}}

% Calligraphic symbols
\newcommand\cB{\mathcal{B}}
\newcommand\cE{\mathcal{E}}
\newcommand\cF{\mathcal{F}}
\newcommand\cO{\mathcal{O}}
\newcommand\cG{\mathcal{G}}
\newcommand\cI{\mathcal{I}}
\newcommand\cP{\mathcal{P}}
\newcommand\cD{\mathcal{D}}
\newcommand\cL{\mathcal{L}}
\newcommand\cU{\mathscr{U}}
\newcommand\cV{\mathscr{V}}
\newcommand\cW{\mathscr{W}}

% Tensors
\newcommand\tI{\mathsf{I}}
\newcommand\tS{\mathsf{S}}
\newcommand\tT{\mathsf{T}}

% Unit vector
\newcommand\ui{\boldsymbol{\hat{\imath}}}

\setlength{\textheight}{8.7in}
\setlength{\columnsep}{2.0pc}
\setlength{\textwidth}{6.6in}
% \setlength{\footheight}{0.2in}
\setlength{\topmargin}{0.05in}
\setlength{\headheight}{0.2in}
\setlength{\headsep}{0.1in}
\setlength{\evensidemargin}{0in}
\setlength{\oddsidemargin}{0in}
% \setlength{\parindent}{1pc}
\setlength{\parindent}{0.0 in}
\setlength{\parskip}{0.1 in}


\title{M.Sc Proposal}
\author{Roman Zeyde}
\begin{document}
\maketitle
\section{Introduction}
Electrophoresis is the motion of dispersed particle relative to a ion-containing fluid,
due to the influence of an electric field. This phenomenon combines electrostatics 
coupled with fluid dynamics, as well as diffusion/advection of ions in the fluid.

The particle (surrounded by fluid) acquire surface charge which is screened by 
a thin layer of ions in the fluid, creating a double-layer structure near the 
particle's surface, called ``Debye layer''.

If the layer width is much smaller than the particle's size
an analytical solution in the Debye-layer can be found \cite{Yariv10}, 
and be used as the appropriate boundary condition for the 
macroscale problem on the particle's surface.

Electrophoresis is described by the following physical phenomena,
using the macroscale equations:
\begin{enumerate}
\item Electrostatic potential - the fluid contains no free charge:
\begin{equation}
    \bnabla \cdot (C \bnabla \varPhi) = 0.
\end{equation}
\item Fluid dynamics - Stokes incompressible flow with electrostatic force component:
\begin{equation}
\left\{ \begin{array}{l}
\bLaplacian \bV - \bnabla P + \Laplacian \varPhi \bnabla \varPhi = 0 \\
\bnabla \cdot \bV = 0 \end{array} \right.
\end{equation}
\item Ions' diffusion and advection - through the fluid's velocity field:
\begin{equation}
\Laplacian C - \alpha \bV \cdot \bnabla C = 0
\end{equation}
\end{enumerate}

The boundary conditions correspond to the specific problem at hand, and
are coupled as well as the partial differential equations themselves.

The closed form approximate solution was developed for spherical particle
and small electric field \cite{Yariv10} but since the equations are coupled
and non-linear, it is hard to extend the analytic solution to more general
settings.

\section{Research goals}
Our goal is to implement an accurate and fast iterative numerical 
solver for the problem described above,
in order to enable the research of cases which has no closed form solution 
(e.g. non-spherical/asymmetric particles, large electric fields).

The solver will be verified against known closed-form solutions 
for each of the sub-problems, as well as the full coupled problem, 
analyzing the discrepancies between the theoretical and numerical results.

Moreover, the convergence rate of the solver will be optimized using 
numerical acceleration methods (e.g. MPE, RRE and Multigrid) to achieve
high accuracy in reasonable solver's execution time.

Achieved results will provide a numeric model for interesting small-scale physical phenomena,
such as an ``self-electrophoretic swimmer'' by numerically modeling an asymmetric particle's configuration.

\section{Achieved results}
The numerical solver is being implemented in MATLAB and verified against
known solution for specific problem settings.

Laplace
Diffusion-advection (upwind)
Stokes
Jacobi, RedBlack, Vanka
Coupling
Spherical
Stress-free




\section{Algorithms and techniques}

The solver applies iterative relaxation to solve the system, by
iterating the equations and solving repeatedly each one of them
for the corresponding variable
(Laplace equation for $\varPhi$, Stokes equation for $(\bV, P)$
and diffusion-advection equation for $C$), where
all other variables keep their values from previous iteration
(somewhat resembling Gauss-Seidel method for linear systems).

Problem's configuration is taken from \cite{Yariv10} and
consists of a spherical particle, surrounded by infinite Stokes fluid.
Suppose that the particle is stationary and fluid flows around it, slipping across
the boundary layer (due to ``Debye layer'' effects) with a velocity field denoted
by $\bV$.

In this setting, together with the usage of spherical coordinates $(R,\theta,\phi)$,
makes the problem axisymmetric around the velocity vector of the fluid (independent
of $\phi$). Thus the problem may be written as two-dimensional partial
differential equation system in $R$ and $\theta$ variables,
such that $R \in [1,\infty)$ and $\theta \in [0, \pi]$.
A natural choice of a grid is logarithmic grid for $R$ and uniform grid for $\theta$.

Thus, each differential operator is written in spherical coordinates on the regular
$(r, \theta)$ grid and being linear -- is represented as a sparse matrix,
making the algebraic computations on the equations much more straightforward.

The boundary conditions are computed for $\theta = 0$ and $\theta = \pi$ using symmetry
considerations. At $R = \infty$, we assume constant fluid velocity
and $\bV = \cal{U} \bzhat$, constant electric field $\bnabla \varPhi = \beta \bzhat$
and constant ionic concentration $C = 1$.
The boundary conditions on $R = 1$ are defined by the boundary layer behavior as described
in \cite{Yariv10}.

ghost points
precond


\begin{thebibliography}{}

\bibitem{Yariv10} E. Yariv, 
``Migration of ion-exchange particles driven by a uniform electric field'',
{\em Journal of Fluid Mechanics, 655 105-121}, 2010.

\end{thebibliography}

\end{document}
