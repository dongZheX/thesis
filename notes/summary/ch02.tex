\section{Numerics}

%%%%%%%%%%%%%%%%%%%%%%%%%%%%%%%%%%%%%%%%%%%%%%%%%%%%%%%%%%%%%%%%%%%%%%%%%%%%%%%%

\subsection{Discretization}
\subsubsection{Spherical Coordinates}
Since the system has axial symmetry, the operators are written in 
spherical coordinates $(r,\theta,\phi)$:
\begin{eqnarray}
\brc{l}{
x = r \sin \theta \cos \phi \\
y = r \sin \theta \sin \phi \\
z = r \cos \theta} 
\end{eqnarray}
Due to axial symmetry, 
so the scalar gradient and the divergence can be written as:
\begin{eqnarray}
\bnabla f &=& \deriv{f}{r}\brhat + \frac{1}{r}\deriv{f}{\theta}\bthetahat \\
\bnabla \cdot \bF &=& \frac{1}{r^2}\deriv{}{r}\pars{F_r r^2 } + 
               \frac{1}{r \sin\theta}\deriv{}{\theta}\pars{F_\theta \sin\theta}
\end{eqnarray}
The scalar laplacian can be written as:
\begin{eqnarray}
\Laplacian f = \bnabla \cdot (\bnabla f)&=& 
 \frac{1}{r^2}\deriv{}{r}\pars{r^2 \deriv{f}{r}} + 
 \frac{1}{r^2 \sin\theta}\deriv{}{\theta}\pars{\sin\theta \deriv{f}{\theta}}
\end{eqnarray}
The vector gradient can be written as:
\begin{eqnarray}
\bnabla \bF &=& \deriv{F_r}{r} \brhat \brhat + \deriv{F_\theta}{r} \brhat \bthetahat + 
\frac{1}{r}\pars{\deriv{F_r}{\theta} - F_\theta} \bthetahat \brhat + 
\frac{1}{r}\pars{\deriv{F_\theta}{\theta} + F_r} \bthetahat \bthetahat
\end{eqnarray}

The vector laplacian can be written as:
\begin{eqnarray}
\bLaplacian \bF &=& 
\left(\Laplacian F_r - \frac{2F_r}{r^2} - 
\frac{2}{r^2 \sin\theta} \deriv{\left(F_\theta \sin\theta \right)}{\theta}\right)\brhat
+ \left(\Laplacian F_\theta - \frac{F_\theta}{r^2 \sin^2\theta} + 
\frac{2}{r^2}\deriv{F_r}{\theta}\right) \bthetahat
\end{eqnarray}

\subsubsection{Grids choice}
In order to discretize the differential operators, 
a regular grid of size $n_r \times n_\theta$ is used:
\begin{eqnarray}
(r_i,\theta_j) &\in& [1, \infty) \times [0,\pi] \\ 
r_i &=& (1+\Delta_r)^i \\
\theta_j &=& \Delta_\theta \cdot j
\end{eqnarray}
where a logarithmic grid is used for $r$ and an uniform one is used for $\theta$.
Note that $r_0 = 1$ and $R_{max} = (1+\Delta_r)^{n_r}$. Thus:
\begin{eqnarray}
\Delta_r &=& \pars{R_{max}} ^ \frac{1}{n_r} - 1 \\
\Delta_\theta &=& \frac{\pi}{n_\theta}
\end{eqnarray}

This grid induces disjoint subdivision of the domain 
$\Omega = [1, R_{max}] \times [0,\pi]$ into cells:
\begin{eqnarray}
\bigcup_{ij}\Omega_{ij} &=& \Omega
\end{eqnarray}
A specific cell $\Omega_{ij}$ and its center $(\bar{r}_i, \bar{\theta}_j)$ are defined by:
\begin{eqnarray}
\Omega_{ij} &=& [r_{i-1}, r_{i}] \times [\theta_{i-1}, \theta_{i}] \\
\bar{r}_i &=& \frac{r_{i-1} + r_{i}}{2} \\
\bar{\theta}_i &=& \frac{\theta_{i-1} + \theta_{i}}{2}
\end{eqnarray}

Each variable is discretized with respect of the cell:
\begin{itemize}
\item $\varPhi$, $C$ and $P$ are represented by their value at the center of each cell, 
using a central grid:
\begin{eqnarray}
\varPhi[i,j] &=& \varPhi(\bar{r}_i, \bar{\theta}_j) \\
C[i,j] &=& C(\bar{r}_i, \bar{\theta}_j) \\
P[i,j] &=& P(\bar{r}_i, \bar{\theta}_j) 
\end{eqnarray}
\item $\bV$ is represented by its values at cell boundaries, using a staggered grid:
\begin{eqnarray}
V_r[i,j] &=& V_r(r_i, \bar{\theta}_j) \\
V_\theta[i,j] &=& V_\theta(\bar{r}_i, {\theta}_j)
\end{eqnarray}
\end{itemize}

\subsubsection{Operator Discretization}
Finite-volume method with linear interpolation between grids is used for flux 
$\boldsymbol{f}$ discretization. 

Denote the following discrete central difference operators:
\begin{eqnarray}
\cD_r(f)[i,j] &=& \frac{f\left[i+\half,j\right] - f\left[i-\half,j\right]}
                       {r\left[i+\half,j\right] - r\left[i-\half,j\right]} \\
\cD_\theta(f)[i,j] &=& \frac{f\left[i,j+\half\right] - f\left[i,j-\half\right]}
					   {\theta\left[i,j+\half\right] - \theta\left[i,j-\half\right]}
\end{eqnarray}

Denote the following interpolation operators:
\begin{eqnarray}
\cI_r(f)[i,j] &=& \frac{
\pars{r\left[i,j\right] - r\left[i-\half,j\right]} 
  f\left[i+\half,j\right] + 
\pars{r\left[i+\half,j\right] - r\left[i,j\right]} 
  f\left[i-\half,j\right] 
}{r\left[i+\half,j\right] - r\left[i-\half,j\right]}
\\
\cI_\theta(f)[i,j] &=& 
\frac{
\pars{r\left[i,j\right] - r\left[i,j-\half\right]} 
  f\left[i,j+\half\right] + 
\pars{r\left[i,j+\half\right] - r\left[i,j\right]} 
  f\left[i,j-\half\right] 
}{r\left[i,j+\half\right] - r\left[i,j-\half\right]}
\end{eqnarray}

The total flux of a cell is equal to zero (due to conservation):
\begin{eqnarray}
\bnabla \cdot \boldsymbol{f} &=& 0 
\\
\frac{1}{r^2} \cD_r\pars{f_r r^2} + 
\frac{1}{r \sin\theta} \cD_\theta\pars{f_\theta \sin\theta} &=& 0 
\\
\cD_r\pars{f_r \cdot r^2 \sin\theta} + \cD_\theta\pars{f_\theta \cdot r \sin\theta} &=& 0 
\end{eqnarray}

Ion flux are discretized on grid cells' boundaries:
\begin{eqnarray}
I_r &=& -\cI_r(C) \cdot \cD_r(\varPhi) \\
I_\theta &=& -\cI_\theta(C) \cdot \frac{\cD_\theta(\varPhi)}{r} 
\end{eqnarray}

Salt flux are discretized on grid cells' boundaries 
(using upwind scheme $\mathcal{U}$ for numerical stability at large cell Peclet number):
\begin{eqnarray}
J_r &=& -\cD_r(C) + \alpha V_r \cdot \mathcal{U}^{\bV}_r (C) \\
J_\theta &=& -\frac{\cD_\theta(C)}{r} + \alpha V_\theta \cdot \mathcal{U}^{\bV}_\theta (C) \\
\nonumber \mathcal{U}^{\bV}_r(C)[i,j] &=& C\left[i-\frac{\sign(V_r)}{2}, j\right] \\
\nonumber \mathcal{U}^{\bV}_\theta(C)[i,j] &=& C\left[i, j-\frac{\sign(V_\theta)}{2}\right] 
\end{eqnarray}

Mass flux $\bV$ is discretized on grid cells boundaries, 
using velocity staggered grid.

Force components are discretized on velocity staggered grid, where
linear interpolation is used for Coulomb force.
\begin{eqnarray}
F_r &=& -\cD_r(P) 
          + \cL(V_r) - \frac{2}{r^2} V_r 
		  - \frac{2}{r^2 \sin\theta} \cI_r(\cD_\theta (V_\theta \sin\theta))
          + \cD_r(\varPhi) \cdot \cI_r(\cL(\varPhi)) \\
F_\theta &=& -\frac{\cD_\theta(P)}{r} 
		  + \cL(V_\theta) - \frac{F_\theta}{r^2 \sin^2\theta} 
		  + \frac{2}{r^2} \cI_\theta(\cD_\theta(F_r))
		  + \frac{\cD_\theta(\varPhi)}{r} \cdot \cI_\theta(\cL(\varPhi)) \\
\nonumber
\cL(f) &=& \frac{1}{r^2}\cD_r\pars{\cD_r(f) r^2} + 
\frac{1}{r^2 \sin\theta} \cD_\theta\pars{\cD_\theta(f) \cdot \sin\theta}
\end{eqnarray}


\subsubsection{Boundary Conditions}
In order to represent discretize problem boundary condtions, 
``ghost'' points method is used. 
Variable grid is extended to include points outside the domain's interior,
so the variables at the ``ghost'' points are set to satisfy 
the discretized boundary conditions, following the analysis below.

The boundary conditions at the ion-exchange boundary ($r=1$) are:
\begin{eqnarray}
\cI_r(\varPhi) &=& -\cI_r(\log C) \\
\cD_r(\varPhi) &=& \cD_r(\log C) \\
V_r &=& 0 \\
V_\theta &=& 4\log\pars{\frac{1 + \exp\left\{\cI_\theta(\zeta)/2\right\}}{2}} \cdot 
			\cD_\theta(\zeta) \\
 \nonumber \zeta &=& - \log \gamma - \cI_r(\varPhi)
\end{eqnarray}
The equations are discretized by using ``ghost'' points, outside the interior of the domain:
\begin{eqnarray}
 \nonumber \frac{\varPhi[0,j] + \varPhi[1,j]}{2} &=& 
	-\frac{\log C[0,j] + \log C[1,j]}{2} \\
 \nonumber \frac{\varPhi[1,j] - \varPhi[0,j]}{\Delta r} &=& 
	\frac{\log C[1,j] - \log C[0,j]}{\Delta r} 
\end{eqnarray}
By adding and subtracting the equations, we have:
\begin{eqnarray}
\varPhi[0,j] &=& -\log\brcs{C[1,j]} \\
C[0,j] &=& \exp\brcs{-\varPhi[1,j]} \\
 \nonumber \zeta[j] &=& \cV -\frac{\varPhi[0,j] + \varPhi[1,j]}{2} 
                     = -\log\gamma -\frac{\varPhi[0,j] + \varPhi[1,j]}{2}
\end{eqnarray}
The slipping conditions can be written by:
\begin{eqnarray}
V_r[0,j] &=& 0 \\
\frac{V_\theta[0,j] + V_\theta[1,j]}{2} &=& V_\theta[j] = 
4\log\pars{\frac{1 + \exp\left\{\frac{1}{2}\frac{\zeta[j] + \zeta[j+1]}{2}\right\}}{2}} 
\cdot \frac{\zeta[j+1] - \zeta[j]}{\bar{\theta}_{j+1} - \bar{\theta}_{j}} \\
 \nonumber V_\theta[0,j] &=& 2 V_\theta[j] - V_\theta[1,j]
\end{eqnarray}

For $\theta = 0$ and $\theta = \pi$, the ``ghost'' points are defined by:
\begin{eqnarray} 
\brc{l}{
\varPhi[i, 0] = \varPhi[i, 1] \\
C[i, 0] = C[i, 1] \\
V_r[i, 0] = V_r[i, 1] \\
V_\theta[i, 0] = 0
} 
\brc{l}{
\varPhi[i, n_\theta+1] = \varPhi[i, n_\theta] \\
C[i, n_\theta+1] = C[i, n_\theta] \\
V_r[i, n_\theta+1] = V_r[i, n_\theta] \\
V_\theta[i, n_\theta+1] = 0
} 
\end{eqnarray}

Far away from the ion-exchange boundary ($r\rightarrow\infty$) are:
\begin{eqnarray}
\frac{\varPhi[n_r + 1, j] - \varPhi[n_r, j]}{\bar{r}_{n_r + 1} - \bar{r}_{n_r}} 
 & = & -\beta \cos\bar{\theta}_j \\
C[n_r + 1, j] & = & 1 \\
V_r[n_r + 1, j] & = & -\cU \cos\bar{\theta}_j \\
V_\theta[n_r + 1, j] & = & \cU \sin\theta_j
\end{eqnarray}
Since the actual grid is finite, $r_{n_r} = R_{max}$ will have an effect on the solution too.

Note that $P$ has no boundary conditions -- so the pressure variable is defined 
on the domain's interior.

%%%%%%%%%%%%%%%%%%%%%%%%%%%%%%%%%%%%%%%%%%%%%%%%%%%%%%%%%%%%%%%%%%%%%%%%%%%%%%%%
\subsection{Solver Design}
\subsubsection{Newton Method}
An operator $\cO$ is defined by its input vector $\bx$ and its output 
$\by = \cO(\bx)$ is defined on a grid $\cG$.

Using an algebra of operators, the composition of operators is well defined:
\begin{eqnarray}
(\cO_1 \circ \cO_2)(\bx) &=& \cO_1(\cO_2(\bx))
\end{eqnarray}
Note that the differentiation of composed operators is computed using chain rule:
\begin{eqnarray}
\bnabla(\cO_1 \circ \cO_2)(\bx) &=& \bnabla \cO_1(\cO_2(\bx)) \cdot \bnabla \cO_2(\bx)
\end{eqnarray}

Thus, in order to solve the equation $\cO(\bx) = \bzero$, a Newton step can 
be computed by:
\begin{eqnarray}
\bzero = \cO(\bx_n + \Delta \bx_n) &\approx& \cO(\bx_n) + \bnabla \cO(\bx_n)\cdot\Delta\bx_n \\
\bnabla \cO(\bx_n) \cdot \Delta \bx_n &=& -\cO(\bx_n) \\
\bx_{n+1} &=& \bx_{n} + \Delta \bx_n \rightarrow
\end{eqnarray}

Therefore, in order to apply Newton method, the residual vector $\cO(\bx)$ and
the gradient matrix of the operator $\bnabla \cO(\bx)$ are to be computed.
Then, a sparse linear system is to be solved to yield current step vector $\Delta \bx$.

When the method converges $\bx_n \rightarrow \bx_\infty$, 
the convergence is quadratic:
\begin{eqnarray}
\|\bx_{n+1} - \bx_\infty\| \le c\|\bx_{n} - \bx_\infty\|^2
\end{eqnarray}
Thus, when the initial guess is close enough to the solution, the solver
will converge in very few steps.


\subsubsection{Operator Representation}
System variable $\bx$ is defined as the concatenation $[\,\varPhi, C, V_r, V_\theta, P\,]$,
taking values in the interior of the problem domain grid.

Thus, each problem variable can be computed by applying an appropriate projection operator:
\begin{eqnarray}
X &=& \cP_X(\bx) \mbox{ for } X \in \left\{\varPhi, C, V_r, V_\theta, P\right\}
\end{eqnarray}

The extended variables $\tilde X$ (with ``ghost'' points values 
from boundary conditions) are be computed, by applying the following non-linear operators:
\begin{eqnarray}
\tilde{\varPhi} &=& \cB_\varPhi(\varPhi, C; \beta) \\
\tilde{C} &=& \cB_C(C, \varPhi) \\
\tilde{V_r} &=& \cB_{V_r}(V_r; \cU) \\
\tilde{V_\theta} &=& \cB_{V_\theta}(V_\theta, C, \varPhi; \cU, \gamma) \\
\tilde{P} &=& P
\end{eqnarray}

The equations of the system (as described above) can be written as non-linear 
operators acting on the extended variables:
\begin{eqnarray}
0 &=& \cE_I (\tilde\varPhi, \tilde C) \mbox{ -- conservation of charge} \\
0 &=& \cE_J (\tilde C, \tilde\bV; \alpha) \mbox{ -- conservation of salt} \\
0 &=& \cE_M (\tilde\bV) \mbox{ -- conservation of mass} \\
0 &=& \cE_{\boldsymbol{p}} (\tilde\bV, \tilde P, \tilde\varPhi) 
\mbox{ -- conservation of momentum}
\end{eqnarray}
The problem can be casted as solving $\cO(\bx) = \bzero$ for $\cO = \cE \circ \cB$,
where $\cB$ is the ``ghost'' point extension operator and $\cE$ is the conservation
equations operator.

\subsubsection{Steady-state}
After convergence, the ``ghost'' points values are updated using $\cB$,
so the force $\tT \cdot \bn$ acting on the ion-exchanger surface
can be computed.

Due to symmetry considerations, the total force $\bF$ is 
aligned with $\ui$ and it vanishes iff $\ui \cdot \bF = 0$.
\begin{eqnarray}
F_\imath = \ui \cdot \bF &=& \oint_\mathcal{S} \ui \cdot \tT \cdot \bnhat  dA = 
\int_0^\pi f_\imath(\theta) \cdot 2\pi \sin\theta d\theta 
\\ \nonumber 
f_\imath &=& \pars{-P + 2\cD_r(V_r) + 
\frac{1}{2}\pars{\cD_r(\varPhi)}^2 - \frac{1}{2r^2}\pars{\cD_\theta(\varPhi)}^2}\cos\theta 
\\ \nonumber 
&& -\pars{\cD_r(V_\theta) - \frac{V_\theta}{r}
+ \frac{1}{r}\cD_r(\varPhi) \cD_{\theta}(\varPhi)}\sin\theta
\end{eqnarray}

The integral above is approximated by 1D numerical quadrature, to yield 
the total force $F_\imath$ as a function of steady-state drift velocity $\cU$.
In order to solve for steady state, Secant method is applied to find
the root of $F_\imath(\cU) = 0$, given the problem parameters $\alpha, \beta, \gamma$. 

\subsubsection{Continuation}

System steady-state drift velocity $\cU(\beta)$ is computed as a function 
of the electric field magnitude for many values of $\beta$.
Since the solution $\bx_\infty$ is continous in $\beta$, the results from previous 
iteration are used as an initial solution $\bx_0$ for the problem of new $\beta$ 
-- helping Newton Method convergence.


%%%%%%%%%%%%%%%%%%%%%%%%%%%%%%%%%%%%%%%%%%%%%%%%%%%%%%%%%%%%%%%%%%%%%%%%%%%%%%%%
\subsection{Solver Implementation}
