\section{Physics}
Electrokinetic phenomena is described by the equations described below.
An ion-exchanger is described by the chemical kinetics at its boundary.
The experiment consists of applying an electic field and measuring the drift velocity of the particles.

%%%%%%%%%%%%%%%%%%%%%%%%%%%%%%%%%%%%%%%%%%%%%%%%%%%%%%%%%%%%%%%%%%%%%%%%%%%%%%%%%%%%%%%%%%%%%%%%%%
\subsection{Microscale}
Denote positive and negative ionic concentration by $c_+$ and $c_-$ respectively.
The respective fluxes are denoted by $\bj_+$ and $\bj_-$.

The electrostatic potential is denoted by $\varphi$. 

The velocity and the pressure field are denoted by $\bv$ and $p$ respectively.

\subsubsection{Nernst-Planck Equations}
The ionic fluxes are given by diffusion (Fick's law), electrostatic forces and advection by the fluid:
\begin{eqnarray}
\bj_+ &=& -\bnabla c_+ - c_+ \bnabla \varphi + \alpha \bv c_+ \\
\bj_- &=& -\bnabla c_- + c_- \bnabla \varphi + \alpha \bv c_-
\end{eqnarray}
Due to conservation of ions, the fluxes are divergence-free:
\begin{eqnarray}
\bnabla \cdot \bj_\pm &=& 0.
\end{eqnarray}
These equation may be re-written by using the following transform:
\begin{eqnarray}
c &=& \frac{c_+ + c_-}{2}\\
q &=& \frac{c_+ - c_-}{2}
\end{eqnarray}
Thus, salt and charge fluxes are defined by:
\begin{eqnarray}
\bj &=& \frac{\bj_+ + \bj_-}{2} = -\bnabla c - q \bnabla \varphi + \alpha \bv c \\
\bi &=& \frac{\bj_+ - \bj_-}{2} = -\bnabla q - c \bnabla \varphi + \alpha \bv q
\end{eqnarray}

These fluxes are also divergence-free:
\begin{eqnarray}
\bnabla \cdot \bj &=& 0 \\
\bnabla \cdot \bi &=& 0 
\end{eqnarray}

\subsubsection{Incompressible Stokes Flow with Electrostatic Force}
Mass conservation (due to constant fluid density):
\begin{eqnarray}
\bnabla \cdot \bv &=& 0
\end{eqnarray}
Momentum conservation (force balance):
\begin{eqnarray}
\Laplacian \bv - \bnabla p + \bnabla \varphi \Laplacian \varphi &=& 0
\end{eqnarray}
Free charge density is given by Poisson equation:
\begin{eqnarray}
\delta^2 \Laplacian \varphi &=& -q
\end{eqnarray}

\subsubsection{Boundary conditions}
Away of the particle, we have uniform flow, uniform electrostatic field
and uniform ionic concentrations:
\begin{eqnarray}
\bv &\rightarrow& -\cU \ui \\
\bE = -\bnabla \varphi &\rightarrow& \beta\ui \\
c_\pm &\rightarrow& 1
\end{eqnarray}

We consider a spherical particle. At its surface ($r=1$ with normal $\bn = \brhat$), 
we have zero slip, anion impermeability, cation selectivity and particle's conductance:
\begin{eqnarray}
\bv & = & \bzero \\
\bn \cdot \bj_- & = & 0 \\
c_+ & = & \gamma \\
\varphi & = & \cV
\end{eqnarray}

%%%%%%%%%%%%%%%%%%%%%%%%%%%%%%%%%%%%%%%%%%%%%%%%%%%%%%%%%%%%%%%%%%%%%%%%%%%%%%%%%%%%%%%%%%%%%%%%%%
\subsection{Separation of scales}
Due to strong electrostatic forces, 
most of the fluid remains electroneutral ($c_+ \approx c_-$), 
except of a thin boundary layer (``Debye layer'') around the ion exchanger. 

\subsubsection  {Bulk-scale equations}
Nernst-Planck equations above for the fluid bulk (outside
the boundary layer) using bulk variables:
\begin{eqnarray}
Q & = & 0 \\
C & = & C_+ = C_- \\
\bJ &=& -\bnabla C + \alpha \bV C \\
\bI &=& -C \bnabla \varPhi
\end{eqnarray}
Since the flow is incompressible ($\bnabla \cdot V = 0$), 
ion conservation results in the following equations:
\begin{eqnarray}
0 = \bnabla \cdot \bJ &\Rightarrow& \Laplacian C - \alpha \bV \cdot \bnabla C = 0 \\
0 = \bnabla \cdot \bI &\Rightarrow& \bnabla \cdot \pars{ C \bnabla \varPhi } = 0
\end{eqnarray}
Stokes flow equations stay the same:
\begin{eqnarray}
\bnabla \cdot \bV &=& 0 \\
\Laplacian \bV - \bnabla P + \bnabla \varPhi \Laplacian \varPhi &=& 0
\end{eqnarray}

\subsection{Debye-scale effective boundary conditions}
We define the rescaled coordinate $\rho$ (normal to ion-exchanger surface), where $\delta \ll 1$:
\begin{eqnarray}
\rho &=& \frac{r-1}{\delta} \\
\end{eqnarray}
Thus, the electric potential and the ionic concentrations are $O(1)$ at Debye layer ($r \approx 1$):
\begin{eqnarray}
\varphi(\rho,\theta) &\sim& \varphi \\
c_\pm(\rho,\theta) &\sim& c_\pm \\
c(\rho,\theta) &\sim& c \\
q(\rho,\theta) &\sim& q
\end{eqnarray}

The radial fluxes at the Debye layer are $O(\delta^{-1})$, due to coordinate rescaling:
\begin{eqnarray}
j_\pm(\rho, \theta) &\sim& \delta^{-1} j_\pm
\end{eqnarray}

The pressure is $O(\delta^{-2})$, due to momentum balance equations:
\begin{eqnarray}
p(\rho, \theta) &\sim& \delta^{-2} p
\end{eqnarray}

The tangential velocity component is $O(1)$ and the radial one is $O(\delta)$.

In order to match bulk $O(1)$ outer radial ionic flux, 
the $O(\delta^{-1})$ inner ionic radial flux $j_\pm(\rho, \theta)$ must vanish:
\begin{eqnarray}
-\deriv{c_\pm}{\rho} \mp c_\pm \deriv{\varphi}{\rho} &=& 0 \\
\deriv{\varphi}{\rho} \pm\frac{1}{c_\pm}\deriv{c_\pm}{\rho} &=& 0 \\
\pars{\varPhi - \varphi} \pm\pars{\log C_\pm - \log c_\pm} &=& 0 \\
\varPhi - \varphi \pm \log \frac{C_\pm}{c_\pm} &=& 0
\end{eqnarray}
This relation can be rewritten as Boltzmann distribution of ionic concentration,
since $C_\pm|_{\rho\rightarrow\infty} = C$:
\begin{eqnarray}
c_\pm = C \exp\left[\pm(\varPhi - \varphi)\right]
\end{eqnarray}

Due to cation selectivity, $c_+|_{\rho=0} = \gamma$.
Denote zeta potential as the voltage drop between the particle surface and the end of
the boundary layer, $\zeta = \cV - \varPhi$. Without loss of generality, we assume that 
$\cV = 0$ thus $\zeta = -\varPhi$.
\begin{eqnarray}
\varPhi + \log\frac{C}{\gamma}&=& 0
\end{eqnarray}
Due to anion impermeability and radial flux continuity:
\begin{eqnarray}
\br \cdot \bj_- = -\deriv{C_-}{r} + C_- \deriv{\varPhi}{r} &=& 0 \\
-\deriv{C}{r} + C \deriv{\varPhi}{r} &=& 0
\end{eqnarray}

The leading-order electric potential in the Debye layer:
\begin{eqnarray}
\deriv{^2 \varphi}{\rho^2} &=& -q = -\frac{c_+ - c_-}{2} = 
-C\frac{\exp\left[+(\varPhi - \varphi)\right] - \exp\left[-(\varPhi - \varphi)\right]}{2} = 
C \sinh(\varphi - \varPhi)
\end{eqnarray}
Integrating with respect to boundary condtions yields:
\begin{eqnarray}
\deriv{\varphi}{\rho} &=& -2\sqrt{C}\sinh\frac{\varphi - \varPhi}{2} \\
\tanh\frac{\varPhi - \varphi}{4} &=& \tanh\frac{\varPhi}{4} e^{-\rho \sqrt{C}}
\end{eqnarray}


Leading-order radial momentum balance in the Debye layer yields:
\begin{eqnarray}
- \deriv{p}{\rho} + \deriv{\varphi}{\rho} \deriv{^2\varphi}{\rho^2} &=& 0
\end{eqnarray}
One integration yields:
\begin{eqnarray}
p &=& \frac{1}{2}\pars{\deriv{\varphi}{\rho}}^2
\end{eqnarray}

Leading-order tangential momentum balance in the Debye layer yields:
\begin{eqnarray}
\deriv{^2 v_\theta}{\rho^2} - \deriv{p}{\theta} 
 + \deriv{\varphi}{\theta} \deriv{^2\varphi}{\rho^2} &=& 0
\end{eqnarray}
By integrating from $\rho = 0$ to $r \rightarrow \infty$, we have Dukhin-Derjaguin slip formula
for tangential velocity component $V_\theta$ (where as the radial component is $V_r = 0$):
\begin{eqnarray}
V_\theta &=& \zeta \cdot \deriv{\varPhi}{\theta} + 
      2\log\pars{1 - \tanh^2 \frac{\zeta}{4}} \cdot \deriv{}{\theta} \log C
\end{eqnarray}

